\begin{abstract}
Reaction networks can be simplified by eliminating
linear intermediate species in partial steady states. In this paper,
we study the question %conjecture of M. Saez, C. Wiuf, and E. Feliu~\cite{saez2015},
whether this rewrite procedure is confluent, so that for any given
reaction network, a unique normal form will be obtained independently 
of the elimination order.
%
We first contribute a counter example %against this conjecture.
which shows that different normal forms of the same network may
indeed have different structures. The problem is that different
``dependent reactions'' may be introduced in different elimination
orders. We then propose a rewrite rule that eliminates
such dependent reactions and prove that the extended rewrite
system is confluent up to kinetic 
rates, i.e., all normal forms of the same network will have the
same structure. However, their kinetic rates may still not be 
unique, even modulo the usual axioms of arithmetics. 
This might seem surprising given that the ODEs of these normal 
forms are equal modulo these axioms. 
%Finally, we present
%an example network from systems biology for the failure of 
%confluence with respect to kinetic rates, that we found in the
%\emph{BioModels} SBML database with an implementation of 
%our rewrite rules. 

%\elisainline{``Searching with the implementation of the rules'' does not sound too well to me, maybe ``analysing'', or some other term\dots

%  If we mention the implementation, could we be asked to publish it with the paper? In that case I would avoid mentioning it\dots
%}
\end{abstract}
