\documentclass[runningheads,a4paper]{llncs}

\usepackage{amsmath,amssymb}
\usepackage[usestackEOL]{stackengine}
\setcounter{tocdepth}{3}
\usepackage{graphicx}
\usepackage{color}
\usepackage{tikz}
\usepackage{url}
\usepackage{wrapfig}
\usepackage{multirow}
\usetikzlibrary{automata}
\usetikzlibrary[arrows,decorations.pathmorphing,backgrounds,positioning,fit,petri]
%\usepackage{mathptmx}
\usepackage{caption}
\usepackage{subcaption}
\captionsetup{compatibility=false}

%\newcommand{\keywords}[1]{\par\addvspace\baselineskip
%\noindent\keywordname\enspace\ignorespaces#1}

%*********for figures************
\newcommand{\initstate}[1]{\stackrel{\!\downarrow}{#1}}
\newcommand{\points}[1]{\marginpar{\qquad \textsf{\small (#1 pts)}}}
\newcommand{\spoints}[1]{\marginpar{\qquad \textsf{\small (#1 pt)}}}
\newcommand{\tab}[1]{\hspace{.2\textwidth}\rlap{#1}}
\newcommand{\bigtab}[1]{\hspace{.3\textwidth}\rlap{#1}}
\newcommand{\itab}[1]{\hspace{0em}\rlap{#1}}

%********for table with fixed width and right align*********
\usepackage{array}
\newcolumntype{L}[1]{>{\raggedright\let\newline\\\arraybackslash\hspace{0pt}}m{#1}}
\newcolumntype{C}[1]{>{\centering\let\newline\\\arraybackslash\hspace{0pt}}m{#1}}
\newcolumntype{R}[1]{>{\raggedleft\let\newline\\\arraybackslash\hspace{0pt}}m{#1}}

\tikzset{
  bend angle=30,
  ->,
  shorten >=1pt,
  node distance=2cm and 2cm,
  on grid,
  auto,
  initial where=above,
  initial text=,
  initial distance=0.6cm,
  inner sep=0.5mm,
  smallstate/.style={circle,draw},
  openstate/.style={},
  succstate/.style={font={$\surd$}},
  bendanglelarge/.style={bend angle=45},
  bendanglesmall/.style={bend angle=10},
  dim/.style={lightgray},
  diag/.style={red},
  bisim/.style={red,dashed,-,bend left},
  loop left/.style={in=195,out=165,loop,swap}
}

%%%%%%%%%%%%%%%%%%%%%%%%%%%%
%for comments
\newcommand\JP[1]{\textcolor{magenta}{#1}}
\newcommand\AM[1]{\textcolor{blue}{#1}}
\newcommand\QY[1]{\textcolor{red}{#1}}
%%%%%%%%%%%%%%%%%%%%%%%%%%%


\begin{document}

\mainmatter
\title{{\sf ASSA-PBN 2.0}: A~Software Tool\\ for Probabilistic Boolean Networks}

\titlerunning{ASSA-PBN 2.0: A~Software Tool for Probabilistic Boolean Networks}

%\author{Andrzej Mizera \and Jun Pang \and Qixia Yuan}
%\authorrunning{Mizera, Pang and Yuan}
%
%\institute{
%Computer Science and Communications, University of Luxembourg, Luxembourg\\
%\email{firstname.lastname@uni.lu}
%}

\author{Andrzej Mizera\inst{1} \and Jun Pang\inst{1,2} \and Qixia Yuan\inst{1}}
\authorrunning{Mizera, Pang, and Yuan}

\institute{
Faculty of Science, Technology and Communication \\
University of Luxembourg, Luxembourg
\and
Interdisciplinary Centre for Security, Reliability and Trust\\
University of Luxembourg, Luxembourg\\
\email{firstname.lastname@uni.lu}
}

\maketitle
%------------------------------------------------------------------------------
\begin{abstract}
We present a~major new release of {\sf ASSA-PBN}, a~software tool for modelling, simulation, and
analysis of probabilistic Boolean networks (PBNs). PBNs are a~widely used computational framework
for modelling biological systems. The steady-state dynamics of a~PBN is of special interest and
obtaining it poses a~significant challenge due to the state space explosion problem which often
arises in the case of large biological systems. In its previous version, {\sf ASSA-PBN} applied
efficient statistical methods to approximately compute steady-state probabilities of large PBNs.
In this newly released version, {\sf ASSA-PBN} not only speeds up the computation of
steady-state probabilities with three different realisations of parallel computing, but also
implements parameter estimation and techniques for in-depth analysis of PBNs, i.e., influence
and sensitivity analysis of PBNs. In addition, a~graphical user interface (GUI) is provided for
the convenience of users.

%\keywords{Probabilistic Boolean networks, Markov chains, steady-state analysis, approximation}
\end{abstract}

%===============================================
\section{Introduction}
\label{sec:intro}
%===============================================
Probabilistic Boolean networks (PBNs)~\cite{SDZ02,TMPTSS13}
are a~modelling framework widely used to model gene regulatory networks (GRNs).
A~PBN model is capable to cope with uncertainties both on the data and model selection levels,
allowing for systematic analysis of global network dynamics in the context of discrete-time Markov chains (DTMCs).
It also provides means for quantifying relative influences and sensitivities of genes in their interactions and for
characterising long-run behaviours of the whole genetic network.
All these analyses are based on the steady-state probability distribution of the associated DTMC.
Therefore, efficient computation of steady-state probabilities is a~crucial foundation for analysing a~PBN.
Due to restricted computational resources,
statistical methods are practically the only viable way to deal with large PBNs.
However, the applicability of existing methods/tools for computing steady-state probabilities of PBNs
are still limited by the network size, e.g., less than 100 nodes~\cite{PAJ14}.
Moreover, to make
the PBN framework generally accepted for mathematical modelling of biological systems,
a~user-friendly tool plays an~important role but currently is still missing.

In this paper, we present a~new release of {\sf ASSA-PBN},
a~tool designed for modelling, simulation and analysis of PBNs.
{\sf ASSA-PBN} in its previous version~\cite{assa} has overcome
the above mentioned network size limitation with the implementation of an~efficient simulator
and state-of-the-art techniques for steady-state analysis, e.g., the two-state Markov chain
approach. The newly released version of {\sf ASSA-PBN} contributes mainly in three aspects.
First, it speeds up the computation of steady-state probabilities of a~PBN by using either
multiple CPU/GPU core based parallelisation or structure-based parallelisation. Second, it
implements parameter estimation and it supports in-depth analysis of PBNs, i.e., long-run
influence and sensitivity analysis of PBNs. Third, it provides a~graphical user interface (GUI) to
ease user interactions with the tool.

A~brief overview of the current functionality of {\sf ASSA-PBN} is presented below.
Items highlighted in bold are the new features added in version 2.0, and
the tool is publicly available at \url{http://satoss.uni.lu/software/ASSA-PBN/}:
\begin{itemize}
\item modelling of PBNs in \textbf{high-level} {\sf ASSA-PBN} format and converting a~model from Matlab-PBN-toolbox format to {\sf ASSA-PBN} format;
\item random generation of PBNs;
\item \textbf{efficient simulation} of a~PBN;
\item computation of steady-state probabilities of a~PBN with either numerical methods
or statistical methods (the two-state Markov chain approach, the Skart method, and the perfect-simulation method)~\cite{MPY15,RP09};
\item \textbf{parallel computation} of steady-state probabilities of a~PBN with either the two-state Markov chain approach
or the Skart method;
\item \textbf{parameter estimation} of a~PBN;
\item \textbf{long-run influence and sensitivity analysis} of a~PBN;
\item a~command-line tool and a~\textbf{GUI}.
\end{itemize}

%===============================================
\section{Preliminaries}
\label{sec:pre}
%===============================================
We briefly introduce the concept of PBN in this section.
PBN models a~system such as a~GRN with binary-valued nodes. For each node there is a~certain
number of Boolean functions, known as \emph{predictor functions}, which determine the value of the
node at the next time step. The selection of the predictor function is governed by a~probability
distribution: a~selection probability parameter is associated with each predictor function of the
node. Two variants of PBNs are considered: \emph{instantaneous PBNs} and \emph{context-sensitive PBNs}. In
the former variant, the selection of a~predictor function is performed for each node at each time
step. In the latter variant, the PBN evolves in accordance with selected predictor functions and
new selection is performed only if indicated by an~additional random variable which is updated at
each time step. Moreover, the so-called PBNs with perturbations allow the system to transit to
the next state due to random perturbations that are governed by a~perturbation rate parameter.
We focus on synchronous PBNs, where the values of all the nodes are updated simultaneously,
while in asynchronous PBNs only one node is updated at a~time step.
The dynamics of a~PBN can be viewed as a~DTMC. In the case
of PBNs with perturbations, the underlying DTMC is \emph{ergodic}, thus having a~unique
\emph{stationary distribution}, the so-called \emph{steady-state} (or \emph{limiting})
\emph{distribution}, which governs the long-run behaviour of the system. Due to the space
limitation, we refer to~\cite{SD10} and \cite[page~4]{TMPTSS13} for a~detailed description of PBNs.

%===============================================
\section{Tool Architecture and New Features}
\label{sec:arc}
%===============================================
\begin{figure}[!t]
\centering
\includegraphics[width=0.85\textwidth]{structure.pdf}
\caption{Architecture of {\sf ASSA-PBN}.}
\label{fig:structure}
%\vspace{-6mm}
\end{figure}

The architecture of {\sf ASSA-PBN} consists of three main modules, i.e., a~modeller, a~simulator, and an~analyser.
The modeller provides a~simple way to construct a~PBN model of a~real-life biological system,
e.g., a~GRN.
The simulator takes the PBN constructed in the modeller and performs
simulation to produce trajectory samples.
Based on the constructed model and generated samples,
the analyser performs basic and in-depth analysis of the PBN.
The analysis results can be used either to interpret the original system or to optimise the
fitting of the system to experimental measurements.
Figure~\ref{fig:structure} depicts the architecture of {\sf ASSA-PBN}.
The analyser requires different input files depending on the analysis task.
While simulator and analyser rely on modeller as input,
the simulation and analysis results can be used to optimise model fitting.

The newly released {\sf ASSA-PBN} preserves the original architecture while making improvements to
all the three modules. We proceed with describing the three modules in more details, while
focusing on the newly implemented features.

%----------------------------------------------------------------------------------
\medskip\noindent
{\bf Modeller.}
%----------------------------------------------------------------------------------
The PBN modeller can either load a~PBN from a~specification file or generate a~random PBN
(e.g., for benchmarking and testing purposes)
complying with a~given parametrisation~\cite{assa}.
In {\sf ASSA-PBN} 2.0, a~high-level PBN definition format is provided and visualisation of a~PBN
is supported in the GUI.
The high-level PBN definition format provides a~way to define Boolean function directly via its
semantics instead of its truth table. The visualisation allows the user to check the details of
each function and to interactively change the values of a~predictor function. The modified PBN can
then be used for the \textit{long-run sensitivity with respect to function perturbation} analysis.

%----------------------------------------------------------------------------------
\medskip\noindent
{\bf Simulator.}
%----------------------------------------------------------------------------------
Statistical approaches are practically the only viable option for the analysis of large PBNs.
However, applications of such methods necessitate generation of trajectories of significant length.
In order to make the analysis to execute in a~reasonable amount of time,
{\sf ASSA-PBN} in its previous version applied the \emph{alias  method}~\cite{WAJ77} to sample the predictor function in each node.
In this new version,
the simulation process is sped up either with the use of the multiple CPU/GPU core parallelisation
technique or with the structure-based parallelisation technique.
The multiple core based technique~\cite{MPY15b} parallelises the simulation with the use of more cores while
the structure-based parallelisation~\cite{MPY16a} achieves the same goal with the use of more memory.
%The structure-based parallel technique analyses the PBN structure to reduce the network size
% and divide node into groups to update multiple nodes simultaneously.
%We refer to~\cite{add technical report} for detailed explanation of this technique.
In order to parallelise the sample generation process,
the algorithms for computing steady-state probabilities in the analyser have to be adjusted.
More details are provided in the analyser section.

Moreover, visualisation of the simulation result is supported in {\sf ASSA-PBN} 2.0 as
well. Time-course evolution of selected node values can be displayed.

%----------------------------------------------------------------------------------
\medskip\noindent
{\bf Analyser.}
%----------------------------------------------------------------------------------
The analyser of {\sf ASSA-PBN} provides three main functionalities:
basic computation of steady-state probabilities of a~PBN,
in-depth computation of long-run influences and sensitivities of a~PBN,
and parameter estimation of a~PBN.
In {\sf ASSA-PBN} 2.0,
the basic computation is largely improved with different parallelisation and multi-core techniques,
while the other two are newly implemented.

\smallskip\noindent
\emph{Parallel computation of steady-state probabilities.}
The steady-state distributions can be computed either in an~exact way or in a~statistical way.
Two iterative methods, i.e., the Jacobi method and the Gauss-Seidel method are implemented for exact computation;
while three statistical methods, i.e., the perfect simulation, the two-state Markov chain approach,
and the Skart method are implemented for the approximate computation~\cite{assa}.
Due to their large memory and time costs,
the two iterative methods and the perfect simulation method are only suitable for analysing small-size PBNs~\cite{assa}.

Based on incremental sampling,
the two-state Markov chain approach and the Skart method are capable of computing steady-state probabilities for large PBNs.
In {\sf ASSA-PBN} 2.0,
we provide two types of techniques to speedup steady-state probability computation.
Firstly, we implement our approach~\cite{MPY15b} of combining
the Gelman \& Rubin method with the two incremental methods
to parallelise steady-state probability computation with multiple CPU/GPU cores.
The Gelman \& Rubin method is used to monitor that all the simulated chains have approximately
converged to the steady-state distribution while the two-state Markov chain approach and the Skart method are used
to determine the sample size required for computing the steady-state probabilities with specified
precision. For a~given precision, the lengths of trajectories used to estimate steady-state
probabilities in the parallel approach are virtually the same as in the original two incremental
methods. However, since the samples are generated with multiple cores in parallel, the
processing time is significantly reduced. Details on the combined algorithms can be found
in~\cite{MPY15b}.
Secondly, we apply our structure-based parallel technique~\cite{MPY16a} to speedup the computation.
This technique contributes in two aspects: reducing the network size by removing irrelevant nodes
and by grouping nodes via merging their predictor functions. The key idea of this technique is to
gain faster simulation speed with the use of larger memory.

\smallskip\noindent
\emph{Long-run influence and sensitivity.}
In a~GRN, it is often important to distinguish which parent gene plays a~major role in regulating
a~target gene and how sensitive the system is with respect to certain changes. PBNs feature
quantification of the importances (formally known as long-run influences) and
sensitivities~\cite{SDZ02,SD10,QD09}. {\sf ASSA-PBN} 2.0 supports computation of long-run
influences and sensitivities, i.e., the long-run influence of a~gene on a~specified predictor
function, the long-run influence of a~gene on another gene, the long-run sensitivity of a~gene in
a~PBN, the long-run sensitivity of a~gene with respect to function perturbation, and the long-run
sensitivity of a~gene with respect to selection probability perturbation. All these
functionalities require the computation of several steady-state probabilities. For each
probability,  a~trajectory of certain length has to be generated. Note that {\sf ASSA-PBN} does
not store the generated trajectory for the sake of memory saving. Instead, {\sf ASSA-PBN} verifies
whether the next sampled state of the PBN belongs to the set of states of interest and stores this
information only. Therefore, a~new trajectory is required when computing the steady-state
probability for a~new set of states of interest. {\sf ASSA-PBN} 2.0 implements computation of
steady-state probabilities of several sets of states in parallel with the two-state Markov chain
approach~\cite{MPY15b}, allowing the reuse of a generated trajectory. The crucial idea is that each
time the next state of the PBN is generated, it is processed for all state sets of interest
simultaneously. Different sets require trajectories of different lengths and the lengths are
determined dynamically through an~iteration process. Whenever the trajectory is long enough for
computing the steady-state probability of a~certain set of states, the steady-state probability of
this set will be computed and this set will not be considered in future iterations.

\smallskip\noindent
\emph{Parameter estimation.}
A~key challenge in constructing a~PBN model is the determination of the model parameters which
make the model match the behaviour of the real system. A~few algorithms~\cite{JR95,GB03} have been
proposed in literature for parameter estimation of biological systems. {\sf ASSA-PBN} 2.0 applies
the \textit{particle swarm optimisation} algorithm for estimating parameters for a~PBN.
This algorithm is an~iterative process for finding an~optimal set of parameters. A~set of
parameters is called a~particle and its fit to the experimental data is verified through a~cost
function. {\sf ASSA-PBN} uses the \textit{mean square error} (MSE) function, i.e.,
$MSE=\frac{1}{n}\sum_{i=1}^{n}(\hat{Y}_i-Y_i)^2$,
where $n$ is the number of measurement data,
$\hat{Y}_i$ is the $i$th measurement data point value,
and $Y_i$ is the computed steady-state probability corresponding to the $i$th data point.
In each iteration, all the particles are updated and verified using the cost function.
The particle that results in the minimum cost function value is the optimal particle.
Particle values are updated based on the current values and the current best optimal particle values
so that each particle is moving towards the direction of the current best optimal particle.
Normally, the verification of the cost function for each particle requires the computation of steady-state probabilities of several sets of states.
To make the computation as fast as possible,
{\sf ASSA-PBN} 2.0 provides the support for computing several steady-state probabilities in parallel
see~\cite{MPY15b} for details).

%===============================================
\section{Evaluation and Case Studies}
\label{sec:evaluation}
%===============================================
As shown in~\cite{assa},
the first version of {\sf ASSA-PBN} has shown a~significant advantage in terms of speed compared
to the Matlab tool optPBN~\cite{TMPTSS13}.
We proceed to compare the performance of {\sf ASSA-PBN} 2.0 with its previous version.
This comparison is done both on randomly generated PBNs and a~PBN constructed for a~real-life
biological system.
The newly released version supports
three types of parallelised computation of steady-state probabilities of a~PBN.
We show in~\cite{MPY15b} that for the multiple CPU core based parallelisation,
the speed-up is approximately linear with the number of cores in our hardware environment (CPU cores up to 40).
%Using 18 randomly generated PBNs,
%we compare the performance of the three types of parallelised computation against the sequential computation.
%For the multiple CPU core based parallelisation, the speed-up is approximately linear with the number of cores in our hardware environment (CPU cores up to 40).
For the multiple GPU core based parallelisation,
{\sf ASSA-PBN} 2.0 can approximately achieve a~speed-up of  200;
while the structure-based parallelisation
shows a~promising performance for sparse networks with large percentage of leaves (for more
details refer to~\cite{MPY16a}).

%\begin{figure}[!t]
%\centering
%\includegraphics[width=0.6\textwidth]{images/speedup.pdf}
%\caption{Performance comparison of parallelised and sequential computation with the two-state Markov chain approach.}
%\label{fig:performance}
%\vspace{-6mm}
%\end{figure}

Moreover, we compared the performance of the three types of parallelisations with the sequential
approach on an~analysis of a~96-node PBN of apoptosis using the two-state Markov chain approach.
In~\cite{MPY15}, the sequential version of the two-state Markov chain approach has been used for
the long-run influence analysis on complex2 from each of its parent nodes. Seven steady-state
probabilities are required to be computed in order to perform the analysis. We re-perform this
computation with the three parallelised versions of the two-state Markov chain approach.
Speed-ups of approximately 200 (GPU), 20 (multiple CPUs) and 10 (structure-based parallelisation)
are obtained with the use of the three different parallel computations.
%We show in Figure~\ref{fig:performance} the gained speed-ups of the three different types of parallelisation techniques comparing to the sequential runs.
Detailed comparison of both the random networks and the real-life case-study can be found at
\url{http://satoss.uni.lu/software/ASSA-PBN/benchmark}.

%===============================================
\section{Future Developments}
\label{sec:future}
%===============================================
First, we plan to implement a~suite of parameter estimation algorithms, e.g., genetic
algorithms. Second, user-friendly improvements, such as support for the standard Systems Biology
Markup Language (SBML) and graphical editing and visualisation of PBN models, will be introduced in the future
releases of {\sf ASSA-PBN} versions.

\medskip\noindent{\bf Acknowledgement.}
Qixia Yuan is supported by the National Research Fund, Luxembourg (grant 7814267).
The authors also want to thank Gary Cornelius for his work on {\sf ASSA-PBN}.

%===============================================
%----------The Bibliography--------------------
%===============================================
\bibliographystyle{splncs}
\bibliography{pbn}

\end{document}
