Osmotic regulation is a hugely important homeostatic system in all cells. Cells respond to osmotic stresses by activating or upregulating proteins involved in the transportation of charged ions, primarily Chlorine, Potassium, Sodium, and Calcium. Additionally, the movement of ions and osmotically obliged water are necessary for many of the cellular hallmarks exhibited in the transformations associated with disease states such as cancer. We present a computable, formal model of the osmotic regulation machinery within a mammalian cell. The model can provide a mechanistic explanation for the behavioural changes observed in highly diverse cellular systems of murine premetastatic Lymph Node stromal cells, and Lung Cancer Fibroblasts. The model explains phenotypic transformations within each cell types, and predicts behaviour from datasets not involved in its generation. Furthermore, we use the model to predict key proteins involved in each transformation, and propose experiments to alter the behaviour of cells in controllable ways.