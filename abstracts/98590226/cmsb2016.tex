\documentclass[a4paper,10pt]{llncs}
\pagestyle{plain}

\input{defs}
\coarsetrue

\begin{document}

\mainmatter  % start of an individual contribution

%opening
% title a bit too long, 
\title{Formal Quantitative Analysis of Reaction Networks 
      Using Chemical Organisation Theory}
\author{Chunyan Mu\inst{1} \and Peter Dittrich\inst{2} 
	\and David Parker\inst{1} \and Jonathan E. Rowe\inst{1}}
\institute{$^1$ School of Computer Science, University of Birmingham\\
	   $^2$ Institute of Computer Science, Friedrich-Schiller-University Jena}

\maketitle

\begin{abstract}
Chemical organisation theory is a framework developed to simplify the
analysis of long-term behaviour of chemical systems. 
An organisation is a set of objects which are closed and self-maintaining. 
In this paper, we build on these ideas to develop novel techniques 
for formal quantitative analysis of chemical reaction networks, 
using discrete stochastic models represented as continuous-time Markov chains. 
We propose methods to identify organisations, 
to study quantitative properties regarding
movement between these organisations and to construct an
organisation-based coarse graining of the model that can be used to
approximate and predict the behaviour of the original reaction network.
\end{abstract}

% \keywords{
% Stochastic reaction networks,
% Probabilistic model checking,\\
% Organisation theory,
% Coarse-graining.
% }

\input{intro}  
\input{prelim}
\input{modelling}
\input{analysis}
\input{eg-model1}
\input{eg-model2}
\input{soundness}
\input{application}
\input{concl}

\subsection*{Acknowledgements}
The authors acknowledge support from the European Union through funding  
under FP7–ICT–2011–8  project  HIERATIC (316705).
We also thank the anonymous reviewers for their helpful and detailed comments.  

\bibliographystyle{plain}
\bibliography{BIB-org}

\input{appendix}
\end{document}
