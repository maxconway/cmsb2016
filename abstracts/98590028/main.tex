%%%%%%%%%%%%%%%%%%%%%%% file main.tex %%%%%%%%%%%%%%%%%%%%%%%%
%
% Article à déposer pour CMSB 2016
%
%%%%%%%%%%%%%%%%%%%%%%%%%%%%%%%%%%%%%%%%%%%%%%%%%


\documentclass[runningheads,a4paper]{llncs}

\usepackage{amssymb}
\setcounter{tocdepth}{3}
\usepackage{graphicx}

\usepackage{url}

%%%%%%%%%%%%%%%%%% Packages %%%%%%%%%%%%%%%%%%%%%%%%%
\usepackage{hyperref}


\usepackage{amsmath}  % Maths
\usepackage{amsfonts} % Maths
\usepackage{amssymb}  % Maths
\usepackage{stmaryrd} % Maths (crochets doubles)
\usepackage{ dsfont }

\usepackage{algorithm}
\usepackage{algpseudocode}

%%% ILP TONY
\usepackage{latexsym}
\usepackage{epic,eepic}
%\usepackage{times}
\usepackage{amsmath}
\usepackage{amsfonts}
\usepackage{multirow}
\usepackage{graphicx}
\usepackage{algorithm}
\usepackage{algpseudocode}
\usepackage{xcolor,colortbl}
\usepackage{ulem}
\usepackage{moresize}
\usepackage{tabularx}
%\usepackage{caption}
%\usepackage{multicol}
%\usepackage{makecell}
%\usepackage{setspace}
%\usepackage[justification=centering]{caption}
%\usepackage{array}
%%%

%%% Marge
\newenvironment{changemargin}[2]{\begin{list}{}{%
\setlength{\topsep}{0pt}%
\setlength{\leftmargin}{0pt}%
\setlength{\rightmargin}{0pt}%
\setlength{\listparindent}{\parindent}%
\setlength{\itemindent}{\parindent}%
\setlength{\parsep}{0pt}%
\addtolength{\leftmargin}{#1}%
\addtolength{\rightmargin}{#2}%
}\item }{\end{list}}

%%%%%%%%%%%%%%%%%%%%
\usepackage{listings}
% Définition du langage ASP
\lstdefinelanguage{ASP}{\^^M}
}

% Définition des styles de tous les listings du document
\lstset{language=ASP,
basicstyle=\small,
columns=flexible,
keywordstyle=\bfseries,
firstnumber=last
}
\renewcommand{\thelstnumber}{\the\value{lstnumber}}
%% fin définition

%%%%%%%%%%%%%%%%%%%%%%%%%%%%%%%%%%
\usepackage{enumerate} % Personnalisation de la numérotation des listes
\usepackage{url}     % Mise en forme + liens pour URLs
\usepackage{array}   % Tableaux évolués
\usepackage{comment}
\usepackage{moresize}
\usepackage{setspace}

\usepackage{prettyref}
\newrefformat{def}{Definition~\ref{#1}}
\newrefformat{fig}{Figure~\ref{#1}}
\newrefformat{pro}{Property~\ref{#1}}
\newrefformat{pps}{Proposition~\ref{#1}}
\newrefformat{lem}{Lemma~\ref{#1}}
\newrefformat{th}{Theorem~\ref{#1}}
\newrefformat{sec}{Section~\ref{#1}}
%\newrefformat{subsec}{Subsect.~\ref{#1}}
\newrefformat{suppl}{Appendix~\ref{#1}}
\newrefformat{eq}{Eq.~\eqref{#1}}
\def\pref{\prettyref}


\usepackage{tikz}
\newdimen\pgfex
\newdimen\pgfem
\usetikzlibrary{arrows,shapes,shadows,scopes}
\usetikzlibrary{positioning}
\usetikzlibrary{matrix}
\usetikzlibrary{decorations.text}
\usetikzlibrary{decorations.pathmorphing}

%\input{macros/macros}

%%%%%
% Macros générales
\def\Pint{\textsc{PINT}}


% Notations spécifiques à ce papier
\newcommand{\PHdirectpredec}[1]{\PHs^{-1}(#1)}
\newcommand{\PHpredec}[1]{\f{pred}(#1)}
\newcommand{\PHpredecgene}[1]{\f{reg}({#1})}
\newcommand{\PHpredeccs}[1]{\PHpredec{#1} \setminus \Gamma}

\tikzstyle{boxed ph}=[]
\tikzstyle{sort}=[fill=lightgray,rounded corners]
\tikzstyle{process}=[circle,draw,minimum size=15pt,fill=white,
font=\footnotesize,inner sep=1pt]
\tikzstyle{black process}=[process, fill=black,text=white, font=\bfseries]
\tikzstyle{gray process}=[process, draw=black, fill=lightgray]
\tikzstyle{current process}=[process, draw=black, fill=lightgray]
\tikzstyle{process box}=[white,draw=black,rounded corners]
\tikzstyle{tick label}=[font=\footnotesize]
\tikzstyle{tick}=[black,-]%,densely dotted]
\tikzstyle{hit}=[->,>=angle 45]
\tikzstyle{selfhit}=[min distance=30pt,curve to]
\tikzstyle{bounce}=[densely dotted,->,>=latex]
\tikzstyle{hl}=[font=\bfseries,very thick]
\tikzstyle{hl2}=[hl]
\tikzstyle{nohl}=[font=\normalfont,thin]


\tikzstyle{aS}=[every edge/.style={draw,->,>=stealth}]
\tikzstyle{Asol}=[draw,circle,minimum size=5pt,inner sep=0,node distance=1.5cm]
\tikzstyle{Aproc}=[draw,node distance=1.2cm]
\tikzstyle{Aobj}=[node distance=1.5cm]
\tikzstyle{Anos}=[font=\Large]

%\tikzstyle{AprocPrio}=[Aproc,double]
\tikzstyle{AsolPrio}=[Asol,double]
\tikzstyle{AprocPrio}=[Aproc,double]
\tikzstyle{aSPrio}=[aS,double]

% Commandes À FAIRE
%\usepackage{color} % Couleurs du texte
%\newcommand{\todo}[1]{\textcolor{red}{\textbf{[[#1]]}}}
%\newcommand{\TODO}{\todo{TODO}}

%%%%%
% Id est
%\newcommand{\ie}{\textit{i.e.} }
\newcommand{\ie}{i.e.\ }
\newcommand{\resp}{resp.\ }

% Césures
\hyphenation{pa-ra-me-tri-za-tion}
\hyphenation{pa-ra-me-tri-za-tions}

%%% A ENLEVER pour les modifications en Rouge
\usepackage{color}
\newcommand{\Emna}[1]{\textcolor{red}{#1}}
\newcommand{\Morgan}[1]{\textcolor{blue}{#1}}
\newcommand{\EmnaRq}[1]{\textcolor{green}{#1}}

% Normal emph
\usepackage{ulem}
\normalem

\input{macros/macros}
\input{macros/macros-ph}
\input{macros/tikzstyles2.tex}
\input{macros/macros-abstr}

%%%%%%%%%%%%%%%%%%%%%%%%%%%%%%%%%%


%\urldef{\mailsa}\path|{emna.ben-abdallah, tony.ribeiro, morgan.magnin, olivier.roux}@irccyn.ec-nantes.fr|    
%\urldef{\mailsb}\path|{inoue}@nii.ac.jp|    
%\newcommand{\keywords}[1]{\par\addvspace\baselineskip
%\noindent\keywordname\enspace\ignorespaces#1}

\begin{document}

\mainmatter  % start of an individual contribution

\title{Inference of Delayed Biological Regulatory Networks from Time Series Data
}


% a short form should be given in case it is too long for the running head
\titlerunning{Inference of DBRNs from time series data}


\author{Emna Ben Abdallah\inst{1} \and  Tony Ribeiro\inst{1}  \and Morgan Magnin\inst{1,2} \and Olivier Roux\inst{1}, \\ and Katsumi Inoue \inst{2}}

\authorrunning{E. Ben Abdallah, T. Ribeiro, M. Magnin, O.Roux, K. Inoue}
% (feature abused for this document to repeat the title also on left hand pages)

% the affiliations are given next; don't give your e-mail address
 %unless you accept that it will be published
\institute{LUNAM Universit\'e, \'Ecole Centrale de Nantes,
 IRCCyN UMR CNRS 6597\\ (Institut de Recherche en Communications et Cybern\'etique de Nantes), \\
  1 rue de la No\"{e}, 44321 Nantes, France. \\
\and
National Institute of Informatics, \\
2-1-2, Hitotsubashi, Chiyoda-ku, Tokyo 101-8430, Japan.
}

%\mailsa\\
%\mailsb

%
% NB: a more complex sample for affiliations and the mapping to the
% corresponding authors can be found in the file "llncs.dem"
% (search for the string "\mainmatter" where a contribution starts).
% "llncs.dem" accompanies the document class "llncs.cls".
%

\toctitle{Inference of delayed biological regulatory networks from time series data}
\tocauthor{Authors' Instructions}
\maketitle
%\vspace{-0.4cm}
\begin{abstract}
The modeling of Biological Regulatory Networks (BRNs) relies on background knowledge, deriving either from literature and/or the analysis of biological observations.
But with the development of high-throughput data, there is a growing need for methods that automatically generate admissible models.
Our research aim is to provide a logical approach to infer BRNs based on given time series data and known influences amoung genes.
In this paper, we propose a new methodology for models expressed through a timed extension of the Automata Networks (well suited for biological systems). The main purpose is to have a resulting network as consistent as possible with the observed datasets. The originality of our work consists in the integration of quantitative time delays directly in our learning approach. 
We show the benefits of such automatic approach on dynamical biological models, the DREAM4 datasets, a popular reverse-engineering challenge, in order to discuss the precision and the computational performances of our algorithm. 

\keywords{inference model, dynamic modeling, delayed biological regulatory networks, automata network, time series data.}
\end{abstract}

  \input{parts/intro}
  \input{parts/automata-network}

  \input{parts/Generation-PH}
  \input{parts/case_study}

  \input{parts/DREAM}
  \input{parts/concl}
%\section*{Acknowledgment}
%This paper is based upon work supported by the Agence Nationale de la Recherche for the HyClock project under Grant No. ANR-14-CE09-0011.
\clearpage
  \bibliographystyle{plain}
  \bibliography{biblio}
%\section*{Appendix: Springer-Author Discount}

\clearpage
\appendix

\section{Appendixes}

\subsection{Proof of Theorem \ref{th:correct}}

	\begin{theorem}[Completeness]% and Correctness]
		Let $\AN=(\ANsort,\ANstate,\ANtrans)$ be a Timed Automata Network, $\Gamma$ be a chronogram of the components of $\AN$, $i \in \mathsf{N}$ and $R \in \ANtrans$ be the set of timed local transitions that realized the chronogram $\Gamma$ such that $(a_i,l,a_j,\delta) \in R \implies |l| \leq i$.
    	Let $\chi$ be the regulation influences of all $a \in \ANsort$.
		Let $\AN'=(\ANsort,\ANstate,\emptyset)$ be a Timed Automata Network.
    	% and $\varphi'$ be the set of timed local transitions of $\AN$ such that $\varphi' \subseteq \varphi$.
		Given $\AN'$, $\Gamma$, $\chi$ and $i$ as input, Algorithm \ref{alg:MoT-AN} is complete:
		it will output a set of Timed Automata Network $\phi$,
		such that $\exists \AN''=(\ANsort,\ANstate,\varphi') \in \phi$ with $R \subseteq \varphi'$.

		\begin{proof}
		Let us suppose that the algorithm is not complete, then there is a timed local transition $h \in R$ that realized $\Gamma$ and $h \not \in \varphi'$.
		In Algorithm \ref{alg:MoT-AN}, after step $1$, $\varphi$ contains all timed local transitions that can realize each change of the chronogram $\Gamma$.
		Here there is no timed local transition $h \in R$ that realizes $\Gamma$ which is not generated by the algorithm, so $h \in \varphi$.
		Then it implies that at step 2, $\forall \varphi', h \not \in \varphi'$.
		But since $h$ realizes one of the change of $\Gamma$ and $h$ is generated at step 1, then it will be present in one of the minimal subset of timed local transitions.
		Such that $h$ will be in one of the networks outputted by the algorithm.
		%It means that there is a time step $t$ where $a$ is applicable, but in the chronogram $C$ there is no gene change at $t$.
		%But since $a \in A$, a change should have occurs at $t$, this is a contradiction.
		%So there is no $a \in A$ that could be remove from $A''$ at step 2, the algorithm is complete.
		$\qed$
		%
		%The algorithm is also correct and this conclusion is trivial from the operation of step 2.
		%If it was not correct, then there is an action $a \in A''$ that is not consistent with $C$.
		%But all such action are removed at step 2.
		%So that the algorithm output is correct.
		%
		%All action of the outputted Process Hitting $PH'$ are consistent with the input chronogram $C$.
		\end{proof}
	\end{theorem}
    
\subsection{Proof of Theorem \ref{th:complexity}}
	
	\begin{theorem}[Complexity]
		Let $\AN=(\ANsort,\ANstate,\ANtrans)$ be a Timed Automata Network,
        $|\ANsort|$ be the number of automaton of $\AN$ and $\eta$ be the total number of local state of a automaton of $\AN$.
		Let $\Gamma$ be a chronogram of the components of $\AN$ over $\tau$ units of time, such that $c$ is the number of changes of $\Gamma$.
		%The complexity of completing $PH$ by generating actions from the observations of $C$ with Algorithm \ref{alg:PHC} belongs to $O(c\cdot i^S + (T-c) \cdot  P \cdot  i^S)$ that is bound by
		%$O(T^2\cdot P^{S+2})$.
		The memory use of Algorithm \ref{alg:MoT-AN} belongs to $O(\tau \cdot i^{|\ANsort|+1} \cdot  2^{\tau\cdot i^{|\ANsort|+1})}$
		that is bounded by $O(\tau \cdot |\ANsort|^{T\cdot  |\ANsort|^{|\ANsort|+1}})$.
		The complexity of learning $\AN$ by generating timed local transitions from the observations of $\Gamma$ with Algorithm \ref{alg:MoT-AN} belongs to
		$O(c\cdot i^{|\ANsort|+1} + 2^{2\cdot \tau \cdot  i^{|\ANsort|+1}} + $c$ \cdot  2^{\tau \cdot i^{|\ANsort|+1}})$,
        that is bounded by $O(\tau \cdot 2^{3\cdot \tau \cdot |\ANsort|^{|\ANsort|+1}})$.
		\begin{proof}
			% Memory
			Let $i$ be the maximal indegree of a timed local transition in $\AN$, $0 \leq i \leq |\ANsort|$.
			Let $p$ be an automaton local state of $\AN$ then $|\ANsort|$ is maximal the number of automaton that can influence $p$.
			There is $i^{|\ANsort|}$ possible combinations of those regulators that can influences $p$ at the same time forming a timed local transition.
			There is at most $\tau$ possible delays,
			so that there are $\tau \cdot |\ANsort| \cdot i^{|\ANsort|}$ possibles timed local transitions,
			thus in Algorithm \ref{alg:MoT-AN} at step 1, the memory is bounded by $O(\tau \cdot i^{|\ANsort|+1})$,
			which belongs to $O(\tau \cdot |\ANsort|^{|\ANsort|+1})$ since $0 \leq i \leq |\ANsort|$.
			Generating all minimal subsets of timed local transitions $\varphi$ of $\AN$ that can realize $\Gamma$ can require to generate at most $2^{\tau \cdot  |\ANsort| \cdot  i^{|\ANsort|+1}}$ set of rules.
			Thus, the memory of our algorithm belongs to $O(\tau \cdot  i^{|\ANsort|+1} \cdot  2^{\tau \cdot  i^{|\ANsort|+1}})$
			and is bounded by $O(\tau \cdot  |\ANsort|^{|\ANsort|+1} \cdot  2^{\tau \cdot  |\ANsort|^{|\ANsort|+1}})$.
			% Run time
			%At each time step, atmost one gene can change its value in the chronogram $C$.
			%Atmost $i^S$ actions can be produced for each gene change, the complexity of step 1 is then $O(c \cdot  i^S)$.
			%In step 2 of Algorithm \ref{alg:PHC} at each time step where there is no gene change, it checks all actions generated so far.
			%Since there is $T \cdot  P \cdot  i^S$ possibles actions, this operation complexity is $O( (T-c) \cdot  T \cdot  P \cdot  i^S)$.
			%Then the complexity of Algorithm \ref{alg:PHC} is $O(c\cdot i^S + (T-c) \cdot  T \cdot  P \cdot  i^S)$.
			%Since $0 \leq i \leq P$ and $0 \leq c \leq T$ the complexity belongs to $O(T\cdot P^S + T^2 \cdot  P^{S+1})$ and is bound by $O(T^2\cdot P^{S+2})$.
		
			%Sampling the chronogram (step 1) is linear in the number of time step and then bound by $O(T)$.
			%Merging the actions at step 2 is polynomial in the number of actions and the complexity of this operation belongs to $O((T\cdot P\cdot i^S)^2)$.
			The complexity of this algorithm belongs to $O(c \cdot i^|\ANsort|+1)$.
			Since $0 \leq i \leq |\ANsort|$ and $0 \leq c \leq \tau$ the complexity of Algorithm \ref{alg:MoT-AN} is bounded by $O(\tau \cdot |\ANsort|^{|\ANsort|+1}))$.
		
			Generating all minimal subsets of timed local transitions $\varphi$ of $\AN'$ that realize $\Gamma$ can require to generate at most $2^{\tau \cdot i^{|\ANsort|+1}}$ set of timed local transitions.
			Each set has to be compared with the others to keep only the minimal ones, which costs $O(2^{2\cdot \tau \cdot i^{|\ANsort|+1}})$.
			Furthermore, each set of timed local transitions has to realize each change of $\Gamma$, it requires to check $c$ changes and it costs $O(c \cdot  2^{\tau \cdot i^{|\ANsort|+1}})$.
			Finally, the total complexity of learning $\AN$ by generating timed local transitions from the observations of $\Gamma$ belongs to
			$O(c\cdot i^{|\ANsort|+1} + 2^{2\cdot \tau\cdot i^{|\ANsort|+1}} + c \cdot  2^{\tau \cdot  i^{|\ANsort|+1}})$.
			that is bounded by $O(3\tau \cdot 2^{2\cdot \tau \cdot |\ANsort|^{|\ANsort|+1}})$.
			%that is bounded by $O(T\times |\ANsort|^{|\ANsort|} + (\tau \times |\ANsort|^{|\ANsort|+1})^2 + 2^{2\times \tau \times |\ANsort|^{|\ANsort|+1}} + \tau \times 2^{\tau \times |\ANsort|^{|\ANsort|+1}})$
			$\qed$
		\end{proof}
	\end{theorem}
    
    \subsection{DREAM4: Influence network}
    
The figure \ref{fig:influence-graph-DREAM4} presents the regulatory graph that we are based on to identify the signs (negative or positive), the thresholds and the quantitative time delays of the learned transitions.
    \begin{figure}[h]\centering
\includegraphics[width =0.6\linewidth]{images/Dream4-100.png}
\caption{The influence network of the DREAM4 challenge model (100 genes) given by GeneNetWeaver (GNW) data generator \cite{schaffter2011genenetweaver}. Each node is a gene and each edge is an influence from the source to the target gene.}
\label{fig:influence-graph-DREAM4}
\end{figure}

\end{document}
