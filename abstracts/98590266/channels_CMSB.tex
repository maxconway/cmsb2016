% This is LLNCS.DOC the documentation file of
% the LaTeX2e class from Springer-Verlag
% for Lecture Notes in Computer Science, version 2.4
\documentclass{llncs}
\usepackage{pgfplots}
\usepackage{subfigure}
\usepackage[margin=1.5in]{geometry}

\usetikzlibrary{external}
\usepgfplotslibrary{external}
\tikzexternalize[prefix=figures/external/]
%\newcommand\tikzsetnextfilename[1]{}

\usepackage{placeins}
\usepackage{llncsdoc}

\usepackage{graphicx}
\usepackage[linesnumbered,tworuled,vlined]{algorithm2e}
\usepackage{amsmath}
\usepackage{amssymb,dsfont}
\usepackage{bm,color}


\DeclareMathAlphabet{\mathonebb}{U}{bbold}{m}{n}
\newcommand{\one}{\ensuremath{\mathonebb{1}}}

%\usepackage{pgf,pgfarrows,pgfnodes,tikz,color,tkz-berge}
%\usetikzlibrary{arrows,shapes,snakes,automata,backgrounds,petri,topaths,calc}



\DeclareMathOperator*{\argmin}{\arg\!\min}

\newcommand{\vect}[1]{\ensuremath{  #1 } }
\newcommand{\ord}[1]{ {\mathcal o}( #1 )}
\newcommand{\Ord}[1]{ {O}( #1 )}
\newcommand{\R}{{\mathbb R}}
\newcommand{\Z}{{\mathbb Z}}
\newcommand{\N}{{\mathbb N}}
\newcommand{\D}[2]{ \ensuremath{ \frac{\mathrm{d} #1 }{\mathrm{d} #2 } }}
\newcommand{\DP}[2]{ \ensuremath{ \frac{\partial #1 }{\partial #2 } } }
\newcommand{\Pred}{\mathrm{Pred}}
\newcommand{\Succ}{\mathrm{Succ}}



%\newtheorem{definition}{Definition}[section]

%\newtheorem{theorem}{Theorem}[section]
%\newtheorem{lemma}[theorem]{Lemma}
%\newtheorem{proposition}[theorem]{Proposition}
%\theoremstyle{definition}
%\newtheorem{definition}[theorem]{Definition}
%\theoremstyle{remark}
%\newtheorem{remark}[theorem]{Remark}


%%%%%%% equation numbering in inline - to conserve space but not too good idea
\makeatletter
\newcommand*{\inlineequation}[2][]{%
  \begingroup
    % Put \refstepcounter at the beginning, because
    % package `hyperref' sets the anchor here.
    \refstepcounter{equation}%
    \ifx\\#1\\%
    \else
      \label{#1}%
    \fi
    % prevent line breaks inside equation
    \relpenalty=10000 %
    \binoppenalty=10000 %
    \ensuremath{%
      % \displaystyle % larger fractions, ...
      #2%
    }%
    ~\@eqnnum
  \endgroup
}
\makeatother

%%%%%%%%%%%%%%%%%%%%%%%%%%%%%%%%%%


\newcommand{\corr}[1]{ {\color{red} \bf #1} }

%\renewcommand{\corr}[1]{ {#1} }

\begin{document}

%%%
\abovedisplayskip=3pt
\belowdisplayskip=3 pt
\abovedisplayshortskip=0pt
\belowdisplayshortskip=0pt
%%%%%

\title{Hybrid Reductions of Computational Models of Ion Channels Coupled
to Cellular Biochemistry}
\author{
Jasha Sommer-Simpson \inst{2}
\and
John Reinitz \inst{2,3}
\and
Leonid Fridlyand \inst{4}
\and
Louis Philipson \inst{4}
\and
Ovidiu Radulescu \inst{1}
}
\institute{
DIMNP UMR CNRS 5235, University of Montpellier, Montpellier, France;\\
\email{ovidiu.radulescu@univ-montp2.fr}
\and
Department of Statistics, University of Chicago, Chicago, USA
\and
Department  of Ecology and Evolution, Department of
Molecular Genetics and Cell Biology, University of Chicago, Chicago, USA
\and
Department of Medicine and Pediatrics, University of Chicago, Chicago, USA
}
\maketitle

\sloppy

\begin{abstract}
Computational models of cellular physiology are often too complex to be analyzed with currently available tools. By model
reduction we produce simpler models with less variables and parameters, that can be more easily simulated and
analyzed. We propose a reduction method that applies to ordinary differential equations models of
voltage and ligand gated ion channels coupled to
signaling and metabolism. These models are used for studying various biological functions such as neuronal and cardiac activity, or insulin production by pancreatic beta-cells.
%We argue that
Models of ion channels coupled to cell biochemistry share a common structure. For
such models we identify fast and slow sub-processes, driving and slaved variables,
as well as a set of reduced models.
Various reduced models are valid locally and can change
on a trajectory. The resulting reduction is hybrid, implying transitions from
one reduced model (mode) to another one.
\end{abstract}

\section{Introduction}

Ion channels are essential in biological
processes that involve fast modifications of cell physiology. They control
the flows and the gradients of ions across the plasma membrane, as well as the membrane potential. Ion channels
can open and close  as a function of the  membrane potential
and/or of the concentrations of ligands such as ATP. The multiple control of ion channels
implies positive and negative feed-back loops that are responsible for rapid bursts and
oscillations of the electrical activity of the cells. Excitability of parts or of entire plasma
membrane allows the generation and  propagation of
action potentials needed for communication between neurons, for muscle
contraction, or for endocrine secretion by specialized cells such as
pancreatic beta cells \cite{hille2001ion,fridlyand2009model,fridlyand2013ion}.
The dynamics of ion channels can be very intricate. Firstly, there are many types of
interacting ion channels, each channel having several subunits that react to voltage and
ligands. Secondly, ion channels are integrated in the cell's physiology and
interact strongly with metabolism and signaling.
For these reasons, models of virtual cells and organs should necessarily include ion channels
especially for functions such as nervous impulse transmission, muscle and cardiac activity \cite{iyer2004computational}, or
insulin secretion by pancreatic islets \cite{fridlyand2016pancreatic}. Biological function is often not a property of
a single cell but an emerging property of interacting cells. Therefore, realistic
models of physiology are necessarily multicellular and should contain hundreds or
thousands of cells that are coupled together electrically and biochemically.
Such models can be computationally expensive if the dynamics of single cells contain
many variables and parameters. Model reduction is useful for
building virtual physiology models that are both realistic and computationally
tractable. By model reduction, one can coarse grain
fast variables dynamics whose computation is expensive, while
keeping accurate descriptions of the dynamics of slower, driving variables.
 {Such a strategy has already been used to model spiral wave dynamics in
cardiac tissue \cite{bueno2008minimal}.}

Several model reduction methods where proposed to reduce ion channel models
\cite{biktasheva2006asymptotic,suckley2003asymptotic,grosu2011cardiac,murthy2012approximate}.
All these methods contain at least one ad hoc stage
in the choice of fast parameters or of small parameters needed for singular
perturbation approximations.
Some attempts to develop automatic model reduction techniques
 based on sensitivity analysis were proposed in \cite{clewley2004dominant,clewley2005computational}.
In previous work, we have developed model reduction methods
for biochemical reactions networks,
based on tropical geometry \cite{NGVR12sasb,radulescu2012frontiers,Noel2013a,soliman2014constraint,radulescu2015symbolic,radulescu2015,samal2015geometric}
allowing the automatic determination of time scales
and of small parameters.
These methods work
for polynomial or rational ordinary differential equations and need to be
extended in order to cope with ion channel models that contain
transcendental functions.

In this paper we propose an extension to ion channels models of reduction methods based on equilibration and
time scales. The concept of equilibration of polynomial dominant terms
used in model reduction by tropical approaches is generalized to situations when
these terms are rational or even transcendental functions. The possibility of
such reductions follows from the property of  ion channels dynamics
to have multiple time scales ranging from milliseconds to minutes (or even to hours in models involving changes of gene expression) \cite{keener1998mathematical}. As is the case for polynomial systems \cite{radulescu2015symbolic},
the time scales of variables are state dependent and can change on a trajectory.
Therefore, a hybrid reduction is appropriate: the coarse
grained dynamics consists in piecewise smooth reduced modes and discrete
transitions between modes. The definition of the
modes and the transitions between them can be justified in the framework of matched
asymptotic expansions from singular perturbations.

 {To obtain such approximations we will employ a mathematical technique called
matched-asymptotic expansion. This singular perturbation technique provides
several approximations, neither of which is uniformly valid, but which have
overlapping domains of validity \cite{lagerstrom1972basic,holmes2012introduction}.
The lack of uniformity could pose problems when the solutions of the full and
reduced models are compared. For instance, the reduction can slightly change the period of
periodic solutions, which leads to large differences at large times.
In order to compare the full and reduced model solutions it is therefore
appropriate to minimize such discrepancies via parameter optimization of the reduced
model. The same method can be used for learning the parameters of the reduced model
from a given data set. }

%Contrary to  standard implementations
%of this technique \cite{holmes2012introduction}, the small parameter  needed for the
%analysis is not given together
%with the problem but can be determined numerically as a ratio of time scales.
%Reductions obtained by matched
%
%We  minimize the discrepancies between
%the trajectories of the full and reduced models  by
%optimization of the parameters of the reduced model.%, using simulated annealing.










\section{Ion channels coupled with cell biochemistry models}



These models contain several types of variables.

{\em Voltage-gated channel variables.} These variables are needed for ion
channel dynamics. They include gating variables as well as the voltage across
the plasma membrane, which controls opening and closing of channels.  The
simplest case is a channel that has only two states: an open state $O$
letting current pass, and a closed state $C$ which allows for no current. For
voltage gated channels, the probabilities per unit time $\alpha, \beta$ that a
channel opens (its state changes from $C$ to $O$) or closes (its state changes
from $O$ to $C$), respectively, are functions of the membrane potential $V$. As
a consequence, the probability $p$ that the channel is open ($p$ also gives the
proportion of open channels) obeys the differential equation
\begin{equation}
\D{p}{t} = \alpha(V) (1-p) - \beta(V) p. \label{eq:channel1}
\end{equation}
Ion channels can have several identical subunits, each of which can be closed
or open. The number of different states for a channel with $m$ identical
subunits is $m+1$ (with each state being represented by a number $k$ between
$0$ and $m$, where $k$ is the number of subunits that are open). The closing and opening of subunits is
modeled as a Markov process with a finite number of states. The Markovian dynamics are
described by a system of ordinary differential equations, known as the
master equation. For a two-state channel, the master equation is
\eqref{eq:channel1}.  When the channels are identical, the master equation has permutation
symmetries. These symmetries can be exploited to obtain
exact reductions of the channels dynamics \cite{keener2009invariant}. As a
result of the exact reduction, the probabilities $p_i$ that $i$ channels are
open ($p_i$ also represents the proportion of channels with $i$ subunits open)
are polynomial functions of a smaller number of variables that satisfy linear
differential equations \cite{keener1998mathematical,keener2009invariant}. For
instance, a channel with two identical subunits has three states corresponding
to $0$,$1$,or $2$ subunits open.  {The corresponding probabilities are derived
from the binomial distribution, namely} $p_0=(1-n)^2$, $p_1=2n(1-n)$ and $p_2=n^2$, where $n$  {is a probability }
satisfying the ordinary differential equation
\begin{equation}
\D{n}{t} = \alpha(V) (1-n) - \beta(V) n, \label{eq:channel2}
\end{equation}
where $\alpha(V),\beta(V)$ are, respectively, the probabilities per unit time that a subunit
opens or closes.
%The rates $\alpha(V),\beta(V)$ follow from the Arrhenius equation and their
%dependence on the voltage can be written as rational combinations of
%exponentials (the hyperbolic tangent is a good approximation for this
%dependence).

Ion currents induce changes of the membrane voltage according to the classical equation
\begin{equation}
C \D{V}{t} = I(t) - \sum_i I_i(t)
\end{equation}
where $I$ is an input  {and
output} current, the variables $I_i$ are currents through through channels of type $i$, and $C$ is the membrane capacitance.

The current-voltage characteristics of a channel of a given type $i$ is affine, the current $I_i$ being proportional
to the difference between voltage $V$ and the rest potential
$V_i$ (defined as the equilibrium voltage corresponding
to the zero current; these depend only on the type of ion and are
the same for different channels of the same ion):
\begin{equation}
I_i = g_i p_i (V - V_i),
\end{equation}
where $g_i$, $p_i$ are, respectively, the open chanel conductance and the proportion
of channels of type $i$ that are open.

{\em Metabolic and signaling variables.} Metabolic and signaling pathways can
be modeled as networks of biochemical reactions. The coupling between
metabolic pathways and ion channel dynamics can be performed  {using the metabolite
concentrations.} For instance,
an increase of the ATP/ADP ratio leads to closure of ATP sensing potassium
channels, which in turn triggers plasma membrane depolarization and opening of
calcium voltage gated channels. This leads to an increase in cytoplasmic
calcium, which triggers important physiological responses such as insulin
release in pancreatic beta cells. A few signaling or metabolic and signaling
variables (e.g. calcium) have contributions to their dynamics coming
from the fluxes fed into and out of the cell by ion channels.
%As a consequence, ion currents contribute to the dynamics of metabolic and
%signaling variables.

In summary, a general model of ion channels coupled to cell biochemistry is described by
the following equations:

\begin{eqnarray}\label{intro:general_model}
\D{V}{t} &=&  (V - V_{\infty})/\tau_V (\vect{h}) \notag \\
\D{h_i}{t} &=& (h_i - h_{i,\infty}(V,\vect{c}))/\tau_{h_i}(V,c) \notag \\
\D{c_k}{t} &=& \sum_{j} S_{kj} R_j (\vect{c}) + \sum_{i \in I(k)} g_{i} p_{i}(h_{i}) (V-V_i) \label{full}
\end{eqnarray}

where $V_{\infty}(\vect{h},\vect{c}) = \sum_i V_i g_i p_i (h_i,\vect{c})$,
$\tau_V(\vect{h},\vect{c}) = \sum_i g_i p_i (h_i,\vect{c})$,
$h_{i,\infty}(V) = \alpha_{i}(V)/(\alpha_{i}(V) + \beta_{i}(V))$, and
$\tau_{h_i}(V) = 1/(\alpha_{i}(V) + \beta_{i}(V))$.
The $R_j$ are multivariate polynomial or rational functions of the metabolite concentrations $\vect{c}$, the
$p_{i}$ are polynomials of the gating variables $h_i$,
the functions $\alpha_{i}(V),\beta_{i}(V)$ are combinations of exponential functions of $V$, the $S_{kj}$
are the entries of the stoichiometric matrix, and the set valued function $I(k)$ denotes the
channels feeding the variable $c_k$.

The voltage $V$ can take negative or positive values but its variations are
bounded $-V_{1} < V < V_2$. The gating variables $h_i$ are also bounded $0 \leq h_i \leq 1$.
The metabolic variables are unbounded, positive concentrations $0 < c_k$; these
variables can have various orders of magnitude, from very small to very large.

\section{Matched asymptotic method.}
The natural framework for approximations of systems with multiple time scales
is the theory of singular perturbations \cite{holmes2012introduction}. The classical presentation of this
theory starts with the identification of a small parameter $\eta$ in the
problem. We then ask: what are the asymptotic behaviors of the solutions when
$\eta \to 0$? In certain cases, this small parameter $\eta$ can be obtained by
nondimensionalization \cite{tang1996simplification}.
Variables and parameters are rescaled via division by
other variables and parameters such that the resulting quotient is unitless. A
smallest parameter can be chosen among unitless parameters. However, the
nondimensionalization can be done in several different ways, and the smallest
unitless parameter is not guaranteed to be the correct small singular
perturbations parameter (extra conditions are needed to guarantee validity of the
approximation). For the time being we suppose that the small parameter has been
identified. We will provide automatic methods to do this in the next sections.

The rescaled equations of a system with fast variables $\vect{x}$ and slow variables $\vect{y}$ read:
\begin{equation}
\D{\vect{x}}{t}  = \frac{1}{\eta} \vect{f}(\vect{x},\vect{y}),
\,\, \D{\vect{y}}{t}  =  \vect{g}(\vect{x},\vect{y}), \label{outer}
\end{equation}
where $\eta$ is a small positive parameter representing the ratio of fast and slow time scales.

The trajectory of a slow/fast system is typically composed of an alternating
sequence of slow parts and fast parts. Let us suppose that the slow segments
are defined by the time intervals $(t_0,t_1)$, $(t_1,t_2)$, $\ldots$. The fast
segments are interlaced between successive slow segments and can be
considered as inner layer solutions placed at $t_1,t_2,\ldots$. In order to
describe the fast inner layer solutions, let us define the variable $\tau =
(t-t_i) / \eta$, $i=1,2,\ldots$. Note that $\tau$ changes by one when $t$
changes by $\eta$, hence $\tau$ corresponds to
short time scales inside the inner layers at $t=t_i$. Using this new time
variable, the differential equations \eqref{outer} become
\begin{equation}
\D{\vect{x}}{\tau}  = \vect{f}(\vect{x},\vect{y}),\,\,
\D{\vect{y}}{\tau}  = \eta  \vect{g}(\vect{x},\vect{y}). \label{inner}
\end{equation}

The \textit{inner layer approximation} is a solution
\begin{align*}X(\tau) &= X_0(\tau) + \eta X_1(\tau) + \ldots,\\
Y(\tau) &= Y_0(\tau) + \eta Y_1(\tau) + \ldots\end{align*}
of the equations \eqref{inner}. We find that, at the lowest order in
$\eta$, the inner layer solution satisfies
\begin{equation}
\D{\vect{X_0}}{\tau}  = \vect{f}(\vect{X}_0,\vect{Y}_0),\,\,
\D{\vect{Y_0}}{\tau}  = 0, \label{inner0}
\end{equation}
which is to say that $Y_0$ is constant.

The \textit{outer layer approximation} is a solution
\begin{align*}
x(t) &= x_0(t) + \eta x_1(t) + \ldots,\\
y(t) &= y_0(t) + \eta y_1(t) + \ldots\end{align*}
of the equations \eqref{outer}. At the lowest order in $\eta$ we find
\begin{equation}
0  = \vect{f}(\vect{x}_0,\vect{y}_0),\,\,
\D{\vect{y_0}}{t}  = g(\vect{x}_0,\vect{y}_0). \label{outer0}
\end{equation}
In other words, in outer layers the fast variables are ``slaved'' by the slow
variables.
At lower order in $\eta$, outer layers are thus described by the
quasi-stationary approximation. Indeed, the outer layer solution lies on a
surface which can be approximated at lowest order by
the equation $f(\vect{x}_0,\vect{y}_0)=0$. Provided that the stability condition
$Re ( Spec( \DP{f}{x}(x_0,y_0) ) ) < 0$ is fulfilled
($\DP{f}{x}$ is the Jacobian matrix
of $f$ with respect to rapid variables $x$) the validity of the quasi-stationarity
approximation is guaranteed by the
Tikhonov theorem  \cite{tikhonov1952systems}.
The surface defined by $f(\vect{x}_0,\vect{y}_0)=0$ also represents
the lowest order approximation of an invariant manifold (low dimensional surface
that contains the reduced dynamics). The existence
of the invariant manifold is guaranteed by Fenichel's results
\cite{fenichel1979geometric}.
An invariant manifold can be stable (attractive)
or unstable; furthermore, the same invariant manifold can have stable parts
that become unstable at bifurcations (for instance of the saddle-node type),
when one or several eigenvalues of  the Jacobian $\DP{f}{x}$ vanish or
touch the pure imaginary axis of the complex plane.
The (slow) outer solutions correspond to dynamics on the same or on different
stable parts of the invariant manifold.
 {We should emphasize that more complex behaviour can occur as a result of the so-called {\em canard} phenomena
when part of the trajectory can lie for some time on the unstable manifold. Such mechanisms, identified in the Morris–-Lecar and FitzHugh–-Nagumo models of excitable systems,
\cite{wechselberger2013canard} will not be discussed here.  }

 {As well known in singular perturbation theory \cite{lagerstrom1972basic},
neither the inner nor the outer solution has uniform validity. However, their domains of validity overlap. As a matter of fact, the two solutions}
must agree for intermediate time scales $(t-t_i) = -\eta^{\alpha}$, $0 < \alpha < 1$.
Hence,
$Y^{(i)}( -\eta^{\alpha-1} ) =  y^{(i)}( t_i - \eta^{\alpha} )$, $X^{(i)}( -\eta^{\alpha-1} ) =  x^{(i)}( t_i - \eta^{\alpha} )$. By taking the limit $\eta \to 0$, we obtain the matching conditions
 \begin{equation}
 \lim_{\tau \to -\infty} Y_0^{(i)}( \tau ) =  y_0^{(i)}(t_i), \,\, \lim_{\tau \to -\infty} X_0^{(i)}( \tau ) =  x_0^{(i)}(t_i), \label{matching1}
 \end{equation}
 where $(Y_0^{(i)},X_0^{(i)})$, and $(y_0^{(i)},x_0^{(i)})$ are the lowest orders of the
 inner and outer layer solutions at $t_i$ and on $(t_{i-1},t_i)$, respectively.
  {For a more formal treatment of this result, the reader may consult \cite{lagerstrom1972basic}.}

 Similarly,
  \begin{equation}
 \lim_{\tau \to \infty} Y_0^{(i)}( \tau ) =  y_0^{(i+1)}(t_i), \,\, \lim_{\tau \to \infty} X_0^{(i)}( \tau ) =  x_0^{(i+1)}(t_i), \label{matching2}
 \end{equation}
where  $(y_0^{(i+1)},x_0^{(i+1)})$ are the lowest orders of the outer layer solutions on $(t_{i},t_{i+1})$.

In other words the boundary conditions of the outer layer solutions are the asymptotic states of the inner layer solutions (corresponding to given fixed values of the slow variables).

A composite lowest order approximation combines inner and outer layers and has to subtract common terms:
\begin{eqnarray}\label{eq:hyb}
y(t) &\sim &\sum [ (y^{(i)}_0(t) - y_0^{(i)}(t_i)) \mathds{1}_{(t_{i-1},t_i]} + Y_0^{i}((t-t_i)/\eta)  ],\notag \\
x(t) &\sim &\sum [ (x^{(i)}_0(t) - x_0^{(i)}(t_i)) \mathds{1}_{(t_{i-1},t_i]} + X_0^{i}((t-t_i)/\eta)  ],
\end{eqnarray}
where $\mathds{1}_{(t_{i-1},t_i]}$ is the indicator function for the interval $(t_{i-1},t_i]$.

 {In this paper we will use only the lowest order of the matched asymptotic expansion. However, higher order
expansions can be also obtained with this method \cite{lagerstrom1972basic,holmes2012introduction}.}

 {The solution \eqref{eq:hyb}} can be seen as a hybrid approximation. The
local dynamics are reduced  {with respect to the full model and, at lowest
order,} are given either by the inner layer (solution of \eqref{inner0}) or by
the outer layer (solution of \eqref{outer0}) approximation.
%Notice that
%the positions $t_i$ of the interior layers are constrained but not entirely
%determined by the matching equations \eqref{matching1},\eqref{matching2}.
%We
%will use positions of bifurcations of the invariant manifold as initial
%estimates for $t_i$, and further refine these values by parameter optimization.
%For the sake of simplicity, we used an unique small scaling
%parameter $\eta$ and slow/fast decomposition for all the slow layers. As a
%matter of fact, both $\eta$ and the slow/fast decomposition can be layer
%dependent.

\section{Algorithm for the hybrid reduction}\label{section:Algorithm}
The matched asymptotic expansion method justifies the possibility of hybrid
approximations but is not a reduction algorithm per se. Several steps, namely
the detection of slow/fast variables and the determination of the inner layer
positions need further development.

\textit{Numerical determination of time scales and outer layers.} In the full (un-reduced) model,
each variable satisfies an ordinary differential equation (ODE) $\D{x_i}{t} = f_i(x_i,\vect{x}^{(i)})$, where  $\vect{x}^{(i)}$
denotes all of the variables other than $x_i$.
Let us denote by $x_i^*(t)$ the solution of $f_i(x_i,\vect{x}^{(i)}(t)) = 0$ that is closest to $x_i(t)$,
where $(x_i(t),\vect{x}^{(i)}(t))$ is a solution of the given ODE system.
We then define two positive indices
\begin{equation}
\begin{aligned}
\tau_i(t) &=  (|f_i(x_i(t),\vect{x}^{(i)}(t))/(x_i(t)-x_i^*(t))|)^{-1}\\
~\text{~and~}~
s_i(t) &= | x_i(t) - x_i^*(t) |/x_{i,s}, \label{indices}
\end{aligned}
\end{equation}
where $x_{i,s}$ is a positive normalizing value (a typical choice is $x_{i,s} = \max | x_i(t) - x_i^*(t) |$).
The index $\tau_i$ is an estimate of the time scale characteristic of the
variable $x_i$. Note that the voltage and gating variables of the
generic ion channel model described by Eqs.~\eqref{full} are $\tau_i =\tau_V$ and
$\tau_i = \tau_{h_i}$, respectively.  The index $s(t)$ is a measure of the
distance between the value of $x_i$ on a trajectory and the imposed value
$x_i^*$ that $x_i$ would have as fast variable in an outer layer solution.
A low value of $s_i$ indicates that $x_i$ is a fast slaved variable in an outer
layer. Variables can be fast, but not slaved, in the inner layers.

\textit{Detection of fast species via sorting of timescales.}
The value $s_i(t)$ is used to determine the intervals of time where fast
variables are slaved; these intervals are the outer layer modes. Suppose that an
outer layer starts at $t_i$ and ends at $t_{i+1}$.  The values $t_i, t_{i+1}$
depend on the trajectory and will change if initial conditions are changed.  We
should therefore look for another way  to define the limits of the outer
layers. A convenient way to do this is to define exit from an outer layer as a condition
on the values of one or several variables of the model, using ordinary
differential equations together with events that trigger transitions between the modes. This is possible
because outer layer solutions belong to invariant manifolds, and the end of a given
outer layer is characterized by loss of stability of the invariant manifold (which
can be written as a condition on the model's variables).

Let us consider that the species time scales are sorted, so that $\tau_1(t) \leq \tau_2(t) \leq \ldots \leq \tau_n(t)$.
The subscripts indexing the values $\tau_i$ may change order, depending on time.
Suppose that, for a given time $t$, there is an index $k$ such that the value
$\eta = \tau_k(t)/\tau_{k+1}(t)$ is much smaller than one.
Then, we can use $k$ to separate fast and slow timescales, and take $\eta$ as the
singular perturbation parameter. Let us note that the multiple timescales
situation $\eta = \tau_k(t)/\tau_{k+1}(t) \to 0$ is covered by the second
theorem of Tikhonov \cite{tikhonov1952systems} and also leads to the outer
layer approximation \eqref{outer0}.

The reduction algorithm is summarized by the following steps:
\begin{enumerate}
\item
    Detect fast species. These have small values of $\tau_i(t)$, separated from
    the rest of the variables by a gap. To be precise,  {choose a number $g$ larger than one. This number $g$ shall be called the \textit{gap width}.}  Define a threshold
    function $\tau_{th}(t)$ such that\\[2pt]
	\null\hspace{3em}\begin{minipage}{0.9\textwidth}
	\begin{enumerate}
	\item $\tau_{th}(t) = (\tau_{k}(t) \tau_{k+1}(t))^{1/2}$,
	\item $\tau_{k+1}(t) / \tau_{k}(t) = \eta^{-1} > g$, and
	\item for each $t$, $k$ is the smallest index satisfying the condition (b).
	\end{enumerate}
	\end{minipage}\\[2pt]
	Then all species such that $\tau_{k}(t) <\tau_{th}(t)$ are declared fast.
\item
    Detect outer layers. These are defined by small values of $s_i$, smaller
    than a fixed threshold.
\item
    Slow, outer layer modes are defined by Equations \eqref{outer0}. Fast, inner
    layer modes are described by Equations \eqref{inner0}.  A slow mode is followed
    and/or preceded by a fast mode.
\item
	Define conditions for exit from outer layers. These conditions depend on
	the values of variables, and can be implemented as ODE system events
	triggering the transition between modes.
\item
	The last step of the algorithm consists in parameter optimization of the
	reduced model by simulated annealing or by other optimization method.
	 {This step is needed for comparison of the full and reduced model
	solutions, or for learning parameters of the reduced model from data.}
\end{enumerate}

\section{A hybrid approximation of the Hodgkin-Huxley  model}


%%%%%%%%%%%%%%%%%%%%%

% Macros and setttings
    %%% TODO: find more accurate values for spikestart and spikeend

    \newif\ifdrawextra
    \drawextratrue

    % Threshold constants
    \def\tauRatio{3.5}
    \def\epsiSlow{0.2}

    % Shortcuts
    \def\VBecomesFast{37.4915}
    \def\VBecomesSlow{38.3835}

    % Sequential timepoints
    \def\Tq{\VBecomesFast}  % V becomes fast
    \def\Tw{37.5295}        % The largest timescale gap falls below the ratio
    \def\Te{37.5455}        % The largest timescale gap rises above the ratio
    \def\Tr{37.627}         % max(slowness_m,slowness_V) falls below the threshold
    \def\Tt{38.0055}        % max(slowness_m,slowness_V) rises above the threshold
    \def\Ty{38.2630}        % max(slowness_m,slowness_V) falls below the threshold
    \def\Tu{\VBecomesSlow}  % V becomes slow

    \def\Yq{0.7021}     % new value of tau_th
    \def\Yr{\epsiSlow}  % the slowness threshold
    \def\Yt{\epsiSlow}  % the slowness threshold
    \def\Yy{\epsiSlow}  % the slowness threshold
    \def\Yu{0.3266}     % old value of tau_th

    %THE BELOW ARE DEPRECATED
    % thresholds
        %\def\epstau{0.2}
        \def\spikestart{\Tq}
        \def\spikeend{\Tu}
        \def\pTone{\spikestart}
        \def\pTtwo{37.625}
        \def\pTthree{37.995}
        \def\pTfour{38.28}
        \def\pTfive{\spikeend}
        \def\eventVone{-51.328}
        \def\eventmone{0.92}
        \def\eventhone{0.1563}
        \def\eventmtwo{0.077341}
        \def\eventhtwo{0.72084}

    % windows bounds
    %\def\spikewindowstart{\spikestart - 0.2}
    %\def\spikewindowend{\spikeend + 0.2}
    \def\spikewindowstart{37.2915}
    \def\spikewindowend{38.5835}
    \def\shortstart{32.5}
    \def\shortend{42.5}

    \tikzset{dashdot/.style={dash pattern=on 1pt off 2pt on 8pt off 2pt}}
    \tikzset{mydotted/.style={very thick, line cap=round, dash pattern=on 0pt off 1cm/13}}
    \pgfplotsset{ every non boxed x axis/.append style={x axis line style=-},
        every non boxed y axis/.append style={y axis line style=-}}


%%%%%%%%%%%%%%%%%%%%%

% Definition of HH model

We have applied our algorithm to several ion channel models. To keep the
presentation short, we illustrate our results on the well-known Hodgkin-Huxley (HH)
model. This model is a four-variable system (\ref{HH:full_ODEs}) of ordinary
differential equations, fitting into the general framework of equations
(\ref{intro:general_model}) above.
    \vspace{5pt}
    \begin{equation}\label{HH:full_ODEs}
      \begin{aligned} \D{V}{t}&=\frac{V-V^*(h_m,h_w,h_n)}{\tau_V(h_m,h_w,h_n)}
      \qquad&\qquad   \D{h_m}{t}&=\frac{h_m-h_m^*(V)}{\tau_{m}(V)}
      \\[10pt]         \D{h_w}{t}&=\frac{h_w-h_w^*(V)}{\tau_{h}(V)}
      &               \D{h_n}{t}&=\frac{h_n-h_n^*(V)}{\tau_{n}(V)},
      \end{aligned}
    \end{equation}
    \\[5pt]
 {where $h_m$ is the sodium channel activation, $h_w$ is the sodium channel inactivation, and
$h_n$ is the potassium channel activation.}
    For each gating variable $x$ in the set $\{m,w,n\}$, the timescale $\tau_x$
    and the imposed value $h_x^*$ are defined by
    \begin{center}
        \hfill $\tau_x(V)=\alpha_x(V)+\beta_x(V)$
        \hfill and
        \hfill $h_x^*(V)=\alpha_x(V)/(\alpha_x(V)+\beta_x(V)),$
        \hfill
    \end{center}
    respectively, where $\alpha_x$ and $\beta_x$ are as given below.
    \vspace{5pt}
    \begin{align*}      \alpha_m (V) &= \frac{0.32(V+54)}{1-\exp(-(V+54)/4)}
        \qquad&\qquad   \beta_m  (V) &= \frac{0.28(V+27)}{\exp((V+27)/5)-1}
        \\[10pt]        { \alpha_w }(V) &= \frac{0.128}{\exp((V+50)/18)}
        \qquad&\qquad   { \beta_w } (V) &= \frac{4}{\exp(-(V+27)/5)+1}
        \\[10pt]        \alpha_n (V) &= \frac{0.032(V+52)}{1-\exp(-(V+52)/5)}
        \qquad&\qquad   \beta_n  (V) &= \frac{0.5}{\exp((V+57)/40)}
    \end{align*}
    \\[5pt]
    The timescale $\tau_V$ and the imposed value $V^*$ for the voltage variable
    are defined below.
    \vspace{0pt}
    %\begin{equation*}
    \begin{gather*} \tau_V(h_m,h_n) =  \frac{ {C}}{g_m  {h_w} h_m^3+g_n h_n^4+g_L},
        \\[10pt]     V^*(h_m,h_n)    =   { \frac{g_m h_w h_m^3V_m+g_nh_n^4V_n+g_LV_L+g_Ih_I}{g_m h_w h_m^3+g_n h_n^4+g_L}} .
    \end{gather*}
    %\end{equation*}
    \\[5pt]
    The conductances $g_x$ are defined by       $ g_m(t) =    100\cdot h_w(t) $%
                                        ,       $ g_n    =    80              $%
                                        ,       $ g_L    =    0.1             $%
                                        , and   $ g_I    =    0.32            $%
                                        .
     {The capacitance was taken to be $C = 1$.}
    Finally, the constants $V_x$ and $h_I$
    are set as follows:         $  V_m = 50  $%
                      ,         $ V_n = -100 $%
                      ,         $ V_L = -67  $%
                      ,  and    $ h_I = 1    $%
                      .
     {The units of   conductance are $mS/cm^2$, those of voltage are $mV$, those of current are $\mu A/cm^2$, and those of capacitance are $\mu F/cm^2$.}
	This model contains only a voltage variable $V$ and three gating variables
	$h_x$, with $x$ in $\{m,w,n\}$. In essence, the equations above are the
	same as those used in the seminal paper of Hodgkin and Huxley
	\cite{hodgkin1952propagation} describing action potentials in the squid
	giant axon; the organization of the equations and parameter values used are
	adopted from \cite{clewley2004dominant}.

 {The analysis and reduction of this model was performed using MATLAB \cite{MATLAB:2013}. }
%%%%%%%%%%%%%%%%%%%%%

\iffalse\FloatBarrier\fi
\subsection{Trajectory of the HH model}

    The system of ODEs described above will quickly converge to a limit
    cycle (see Figure \ref{HH-fig-long-trajectory}).
    We will see that the voltage $V$ is slow except for during the spikes.
    Figure \ref{HH-fig-trajectory} plots all four variables over one period of
    this limiting cycle.

	\begin{figure}[h!]
	\centering
			\input{figures/long_trajectory}
		\caption{Above, left: A plot of the voltage variable $V$ from the (non-reduced)
					HH model, obtained by forward numerical simulation of
					the equations (\ref{HH:full_ODEs}). With parameters as above, this
					limiting cycle has an approximate period of $\Delta t\approx
					49.5$.}
	\end{figure}


	\begin{figure}[h!]
		\subfigure{
			\label{HH-fig-long-trajectory}
			\input{figures/trajectory}
			\label{HH-fig-trajectory}
		}%
		~
		\subfigure{
    \def\thisfigxmin{\shortstart}
    \def\thisfigxmax{\shortend}
        \input{figures/short_imposed}
        \label{HH-fig-short-imposed}
		}%
		\caption{					Above, left:
					One period of the limiting cycle for the HH model.
					The voltage variable is plotted on the upper axes, the middle
					axis displays the trajectories of the gating variables, and
					the timescales $\tau_x$ are plotted on the lower axis. A
					larger timescale means that the given variable $x$ is slower
					to follow its imposed value $x^*$.
					Above, right: A comparison of each variable $x_i$ from the (non-reduced)
                 HH model with its imposed value $x_i^*$, in a
                 short window of time surrounding the spike. The imposed
                 value of $x$ is given by the steady
                 state of the equation
                 $\D{x_i}{t} = f_i(x_i,x^{(i)})$ with $x^{(i)}$
                 held fixed.
					}
	\end{figure}



%%%%%%%%%%%%%%%%%%%%%

\iffalse\FloatBarrier\fi
\subsection{Determination of the imposed values}

It is convenient that in the HH model, the imposed value $x_i^*$ of
each variable $x_i$ can be found as a closed-form function of the variables
other than $x_i$, which are collectively denoted $x^{(i)}$. A comparison of
each variable's trajectory with that of its imposed value can be found in
Figure \ref{HH-fig-short-imposed}.

In the case of a more complex model, it may be possible to find some of the
imposed values $x_i^*$ via computer algebra methods (this is the usual case,
where $f_i$ is a polynomial in $x_i$ having small degree).  {Alternatively, the
computation of the imposed values can be performed via numerical solution of the equation
$0=f_i(x_i^*,x^{(i)})$. In the case of multiple solutions to this equation, we choose the
solution which provides the smallest index $s_i$ and which satisfies the
physical variables constraints, i.e. whose chemical concentrations are positive
and whose gating variables fall in the interval $[0,1]$. }
%\marginpar{Maybe include something about Tikhonov's
%theorem theorem in the references?}

%    \begin{figure}[h!] \centering
%        \input{figures/short_imposed}
%        \caption{A comparison of each variable $x_i$ from the (non-reduced)
%                 Hodgkin-Huxley model with its imposed value $x_i^*$, in a
%                 short window of time surrounding the spike. The imposed
%                 value of $x$ is given by the steady
%                 state of the equation
%                 $\D{x_i}{t} = f_i(x_i,x^{(i)})$ with $x^{(i)}$
%                 held fixed.
%                }
%        \label{HH-fig-short-imposed}
%    \end{figure}

%%%%%%%%%%%%%%%%%%%%%

\iffalse\FloatBarrier\fi
\subsection{Detection of fast species}
According to Eq.\eqref{indices}, the timescale $\tau_{x_i}$  is defined as the
absolute value of the ratio $\frac{x_i^*-x_i}{dx_i/dt}$.  Intuitively,
$\tau_{x_i}$ is an amount of time required for the variable $x_i$ to move
``significantly'' in the direction of its imposed value $x_i^*$. Indeed, if we
hold constant the distance $x_i^*-x_i$ between $x_i$ and its imposed value,
then the timescale $\tau_{x_i}$ varies inversely with the derivative
$\D{x_i}{t}$.  Thus the ``fast'' variables are those $x$'s such that $\tau_x$
is smaller than some given threshold, whereas the ``slow'' variables are the
variables $x$ whose timescales $\tau_x$ are large.

	The timescale threshold $\tau_{th}(t)$, which distinguishes between the
	slow and fast variables, will be allowed to change with time.  Following
	the algorithm in Section \ref{section:Algorithm}, we assign a number
	between 1 and 4 to each of the timescales $\tau_V(t)$, $\tau_m(t)$,
	$\tau_w(t)$ and $\tau_n(t)$, so that the inequality $$\tau_1(t) \leq
	\tau_2(t) \leq \tau_3(t) \leq \tau_4(t)$$ is satisfied for each point in
	time.  At a given point $t$, we write $k$ for the smallest integer such
	that $1\leq k\leq4$ and such that the ratio
	$\frac{\tau_{k+1}(t)}{\tau_k(t)}$ is larger than a chosen value $g$; we
	have chosen $g=\tauRatio$.  Thus, $\tau_k$ is the slowest of the fast
	variables, and $\tau_{k+1}$ is the fastest among the slow variables.  The
	threshold $\tau_{th}$ is defined as the geometric mean
	$(\tau_k(t)\tau_{k+1}(t))^{1/2}$, so that a variable $x$ is fast if and
	only if $\tau_x$ is smaller than $\tau_{th}$.

    See the top axis of Figure \ref{HH-fig-slowness} for a plot of the
    timescales $\tau_x$ and the timescale threshold $\tau_{th}$. Note that the
    variable $h_m$ is always fast, the variables $h_w$ and $h_n$ are always
    slow, and the voltage variable $V$ is fast only during the spike (from
    $t=\VBecomesFast$ to $t=\VBecomesSlow$).
    % In the future, it may be fruitful to consider stratification of variables
    % into a heirarchy of speeds, e.g. ``slow'', ``medium'', and ``fast''.

%%%%%%%%%%%%%%%%%%%%%

\iffalse\FloatBarrier\fi
\subsection{Detection of outer layers}

    The slowness index $s_x$ of each variable $x$ is obtained  by normalizing
    the difference between $x_i$ and $x_i^*$, as per Eq.\eqref{indices}.
 %       $$s_i=\frac{|x_i^*-x_i|}{\max_t|x_i^*(t)-x_i(t)|}.$$
    The bottom axis of Figure \ref{HH-fig-slowness} displays a plot of the
    slowness indices $s_V$ and $s_m$ corresponding to fast variables $V$ and
    $h_m$, respectively. We have chosen $s_{th}=\epsiSlow$ as the threshold for
    distinguishing between slaved and unslaved fast variables: say that $x_i$ is
    slaved at time $t$ if and only if the inequality $s_i(t) \leq s_{th}$ is
    satisfied.

    The position of the outer layer solution is characterized by small values
    $s_i$ for each of the fast variables $x_i$. Referring to the solution of the
    full HH model in Figure \ref{HH-fig-slowness}, the outer layer
    occurs from {\Tr} to {\Tt} and from {\Ty} to {\Tu} (during which times both
    $V$ and $h_m$ are fast), as well as before {\Tq} and after {\Tu} (during
    which times only $h_m$ is fast). See Table \ref{table-of-modes} for a
    summary of information concerning the different modes (inner and outer) as
    well as the transitions between them.

 {At this point we can compare our results with similar approximations of the
HH model. Like \cite{suckley2003asymptotic}, we identify the variable
$h_m$ as fast everywhere. However, contrary to our approach,
\cite{suckley2003asymptotic} considers that $h_m$ is slaved everywhere, which is not
true at least for the parameters values that we use (similar to theirs).  Thus,
with respect to more conventional singular perturbations methods,
our approach has two advantages:
it detects automatically which type of approximation should be applied  and
it considers the possibility of inner layers where fast variables are not slaved.

The approach used in \cite{clewley2004dominant,clewley2005computational} identifies slow and
fast regimes which are equivalent to our outer and inner layers, respectively.
However, the method in \cite{clewley2004dominant,clewley2005computational}  is based on a sensitivity study
of the inputs of each variable rather than on direct testing of quasi-stationarity as in our
approach. The former method requires a complex heuristic to consolidate the results, whereas in our case
the mode decomposition is simply controlled by the two thresholds $g$ and $s_{th}$.
 We therefore expect the method presented in this paper to be better
terms of robustness and precision of the approximation. Indeed, for the HH model,
\cite{clewley2005computational} finds six regimes, and the agreement between trajectories
simulated with the full and hybrid model is only qualitative. }




    \def\thisfigxmin{\spikewindowstart}
    \def\thisfigxmax{\spikewindowend}
    \def\plotheight{\textwidth/4}
    \begin{figure}[h!] \centering
        \input{figures/slowness}
		\caption{Delineation of modes of the (non-reduced) HH
			model.  The top axes show timescales $\tau_{x_i}$ of each of the
			variables in a short window of time surrounding the voltage spike.
			The dotted line (marked as $\tau_{th}$ in the legend) is the moving
			threshold used for distinguishing slow variables from fast ones.
			This timescale threshold $\tau_{th}$ is calculated as the geometric
			mean $(\tau_{k}(t)\tau_{k+1}(t))^{1/2}$ of the smallest two
			adjacent timescales whose ratio is larger than a chosen constant
			value $g$.  The vertical dotted lines mark time-points where
			variables change between slow and fast or between slaved and
			unslaved.
        }\label{HH-fig-slowness}
    \end{figure}

    \begin{table}[h]
    \begin{small}
    \begin{center}

    \begin{tabular}{|c|l|l|l|l|l|l|}
    \hline
    \multicolumn{1}{|p{1.2cm}|}{{Region}} &
    \multicolumn{1}{|p{1.2cm}|}{{Type  }}&
    \multicolumn{1}{|p{1.2cm}|}{{V}}&
    \multicolumn{1}{|p{1.5cm}|}{{$h_m$}} &
    \multicolumn{1}{|p{1.5cm}|}{{$h_w$}} &
    \multicolumn{1}{|p{1.5cm}|}{{$h_n$}}&
    \multicolumn{1}{|p{1.5cm}|}{{ Exit event}}
    \tabularnewline
    \hline
    \hline
    1 & O & slow & fast/slaved & slow & slow & $V>V_1$ \tabularnewline
    \hline
    2 & I & fast/unslaved & fast/unslaved & slow & slow & $m>m_1$ \tabularnewline
    \hline
    3 & O & fast/slaved & fast/slaved & slow & slow & $h<h_1$ \tabularnewline
    \hline
    4 & I & fast/unslaved & fast/unslaved & slow & slow & $m<m_2$ \tabularnewline
    \hline
    5 & O & fast/slaved & fast/slaved & slow & slow & $h>h_2$ \tabularnewline
    \hline
    \end{tabular}

    \def\sep{\hspace{2mm}}
    \begin{tabular}{|l l @{\sep} | @{\sep} l @{\sep} | @{\sep} l @{\sep} l|}
     \hline&&&&\\[-3mm]
               Region    & 1
                        & $h_m = h_m^*$
                        & \parbox{0.2\textwidth}{$\displaystyle\dot x=(x-x^*)/\tau_x$}
                        & for $x\in\{V,h_w,h_n\}$
     \\[3mm]\hline&&&&\\[-3mm]
     Regions   & 2 and 4
                        & $h_w$ and $h_n$ constant
                        & \parbox{0.2\textwidth}{$\displaystyle\dot x=(x-x^*)/\tau_x$}
                        & for $x\in\{V,h_m\}$
     \\[3mm]\hline&&&&\\[-3mm]
     Regions   & 3 and 5
                        & $h_m = h_m^*$ and $V = V^*$
                        & \parbox{0.2\textwidth}{$\displaystyle\dot x=(x-x^*)/\tau_x$}
                        & for $x\in\{h_w,h_n\}$
     \\[3mm]\hline
    \end{tabular}

    \end{center}
    \end{small}
    \caption{\small Summary of the regions and ODEs defining the modes of the hybrid simplification
    of the HH model. The type I stands for inner layer and O for outer layer. Slaved
    variables are set to their quasi-stationary values. Unslaved variables follow ODE dynamics.
    In inner layers slow variables are constant at lowest order approximation
    and fast variables are unslaved. In outer layers fast variables are slaved and slow variables
    are unslaved.}
    \label{table-of-modes}
    \end{table}

%%%%%%%%%%%%%%%%%%%%%

\iffalse\FloatBarrier\fi
\subsection{Differential equations of modes,  {exit events and parameter optimization}}
%  In practice, it is not possible to use the slowness index to trigger
%    transitions from slaved to unslaved modes of the reduced model; because $x$
%    is equal to $x^*$ whenever $x$ is slaved, the slowness index
%    $s=\frac{|x^*-x|}{\max_t|x^*(t)-x(t)|}$ is uniformly zero in those regions
%    where $x$ is slaved. This paper instead adopts the strategy of
 As  discussed in Section \ref{section:Algorithm}, triggering transitions
 between modes of the reduced model is performed via thresholds, listed under
 ``Exit event'' in Table \ref{table-of-modes}. For example, the event
 ``$V>V_1$'' is used to trigger exit from region 1, meaning that there is a
 threshold $V_1$ such that exit from region 1 is signaled by the value of $V$
 surpassing that of $V_1$.
  {For each region, the thresholded variable was chosen from the {\em
 active} ones, i.e. slow variables trigger exit from outer layers and fast
 variables trigger exit from inner layers. If several variables are active at a
 given time, we chose the variable whose logarithmic time derivative has the
 largest absolute value.  The thresholds were refined in order to minimize the
 $L^2$ distance between the trajectories generated by the full and hybrid
 models.  Furthermore, the period of the reduced model was adjusted to match
 that of the full model; this was achieved by adding $C$ to the list of
 optimized parameters.  Local optimization using the  function {\em lsqnonlin}
 of MATLAB (trust-region-reflective least square algorithm) was enough to
 obtain a good fit.}
     After optimization, we arrived at the
    following parameter values: $V_1 = -46.99973,$
                               $m_1 =   0.99496,$
                               $h_1 =   0.03597,$
                               $m_2 =   0.16298,$
                               $h_2 =   0.72711,$
                               $C = 1.00117$.
%Furthermore, the period of the reduced model was changed to match that of the
%full model by increasing the value of $C$ from $1$ to $1.00117$.
For
completeness, the equations governing the model's evolution in each of the five
regions are detailed in Table \ref{table-of-modes}.
%    \def\sep{\hspace{2mm}}
%    \begin{center}
%    \begin{tabular}{l l @{\sep} | @{\sep} l @{\sep} | @{\sep} l @{\sep} l}
%     \hline&&&&\\[-3mm]
%               Region    & 1
%                        & $h_m = h_m^*$
%                        & \parbox{0.2\textwidth}{$\displaystyle\D{x}{t}=\frac{x-x^*}{\tau_x}$}
%                        & for $x\in\{V,h_w,h_n\}$
%     \\[3mm]\hline&&&&\\[-3mm]
%     Regions   & 2 and 4
%                        & $h_w$ and $h_n$ constant
%                        & \parbox{0.2\textwidth}{$\displaystyle\D{x}{t}=\frac{x-x^*}{\tau_x}$}
%                        & for $x\in\{V,h_m\}$
%     \\[3mm]\hline&&&&\\[-3mm]
%     Regions   & 3 and 5
%                        & $h_m = h_m^*$ and $V = V^*$
%                        & \parbox{0.2\textwidth}{$\displaystyle\D{x}{t}=\frac{x-x^*}{\tau_x}$}
%                        & for $x\in\{V,h_w,h_n\}$
%     \\[3mm]\hline
%    \end{tabular}
%    \end{center}
    A juxtaposition of the trajectories of the reduced and full models can be found
    in Fig.~\ref{reduced-trajectory-faraway}.
    % and
    %\ref{reduced-trajectory-closeup}.
    \begin{figure}[h!] \centering
        \includegraphics[width=0.94\textwidth]{figures/figurex.png}
        \caption{The trajectory of each of the four variables of the full and
         reduced HH models.}
        \label{reduced-trajectory-faraway}
%        \includegraphics[width=0.94\textwidth]{figures/figurey.png}
%        \caption{A close-up of the spike in the Hodgkin-Huxley model, comparing
%         the trajectories of the full and reduced model for each of the four
%         variables.}
%        \label{reduced-trajectory-closeup}
    \end{figure}

%\FloatBarrier

%\section{A more complex example}
%
%An example with metabolic variables: bursts and oscillations of calcium. TO DO: results can be presented
%as supplementary material.

\section{Conclusion and future work}
We have shown how to obtain hybrid reduced models for differential equations models of
ion channels dynamics. These hybrid reductions can be used as simplified units of
multiscale models of tissues or organs. In certain cases, hybrid simplifications
can relate biochemical parameters to physiological properties analytically.
For instance, a matched asymptotic simplification of the HH model with
one dimensional description of the slowest outer layer (coarser than the one presented here)
can be used to find an approximate analytic expression relating the period of bursting to
the model parameters. The details of this application will be presented elsewhere.

In this paper we have presented a trajectory based method for reduction. This method has the
advantage of generality and simplicity of implementation, but could, in certain situations,
provide a reduction that is valid only locally in the phase and parameter spaces.
Tropical geometry approaches, currently applied to
polynomial and rational differential equations, do not use trajectory simulations
and their robustness is guaranteed by replacing positive real numbers by orders
of magnitude, i.e. valuations. In this work we borrowed equilibration ideas from tropical methods
but we have not used orders yet. The main difficulty in computing orders
 is the transcendental nature of some voltage dependent terms.
 This will be overcome in future work by using an elimination method in which
valuations are computed as a function of voltage (considered as a parameter).

















\paragraph{Acknowledgments}
This work was supported by the University of Chicago and by the FACCTS (France
and Chicago Collaborating in The Sciences) program.  The authors express their
gratitude to the reviewers for their many helpful comments.

\begin{thebibliography}{10}
\providecommand{\url}[1]{\texttt{#1}}
\providecommand{\urlprefix}{URL }

\bibitem{biktasheva2006asymptotic}
Biktasheva, I., Simitev, R., Suckley, R., Biktashev, V.: Asymptotic properties
  of mathematical models of excitability. Philosophical Transactions of the
  Royal Society of London A: Mathematical, Physical and Engineering Sciences
  364(1842),  1283--1298 (2006)

\bibitem{bueno2008minimal}
Bueno-Orovio, A., Cherry, E.M., Fenton, F.H.: Minimal model for human
  ventricular action potentials in tissue. Journal of theoretical biology
  253(3),  544--560 (2008)

\bibitem{clewley2004dominant}
Clewley, R.: Dominant-scale analysis for the automatic reduction of
  high-dimensional ode systems. ICCS 2004 Proceedings. New England: Complex
  Systems Institute  (2004)

\bibitem{clewley2005computational}
Clewley, R., Rotstein, H.G., Kopell, N.: {A computational tool for the
  reduction of nonlinear ODE systems possessing multiple scales}. Multiscale
  Modeling \& Simulation  4(3),  732--759 (2005)

\bibitem{fenichel1979geometric}
Fenichel, N.: Geometric singular perturbation theory for ordinary differential
  equations. Journal of Differential Equations  31(1),  53--98 (1979)

\bibitem{fridlyand2009model}
Fridlyand, L.E., Jacobson, D., Kuznetsov, A., Philipson, L.H.: {A model of
  action potentials and fast Ca 2+ dynamics in pancreatic $\beta$-cells}.
  Biophysical journal  96(8),  3126--3139 (2009)

\bibitem{fridlyand2013ion}
Fridlyand, L.E., Jacobson, D.A., Philipson, L.: {Ion channels and regulation of
  insulin secretion in human $\beta$-cells: a computational systems analysis}.
  Islets  5(1),  1--15 (2013)

\bibitem{fridlyand2016pancreatic}
Fridlyand, L.E., Philipson, L.H.: {Pancreatic Beta Cell G-Protein Coupled
  Receptors and Second Messenger Interactions: A Systems Biology Computational
  Analysis}. PloS one  11(5),  e0152869 (2016)

\bibitem{grosu2011cardiac}
Grosu, R., Batt, G., Fenton, F.H., Glimm, J., Le~Guernic, C., Smolka, S.A.,
  Bartocci, E.: From cardiac cells to genetic regulatory networks. In:
  International Conference on Computer Aided Verification. pp. 396--411.
  Springer (2011)

\bibitem{hille2001ion}
Hille, B.: Ion channels of excitable membranes. Sinauer Sunderland, MA (2001)

\bibitem{hodgkin1952propagation}
Hodgkin, A., Huxley, A.: Propagation of electrical signals along giant nerve
  fibres. Proceedings of the Royal Society of London. Series B, Biological
  Sciences pp. 177--183 (1952)

\bibitem{holmes2012introduction}
Holmes, M.H.: Introduction to perturbation methods, vol.~20. Springer Science
  \& Business Media (2012)

\bibitem{iyer2004computational}
Iyer, V., Mazhari, R., Winslow, R.L.: A computational model of the human
  left-ventricular epicardial myocyte. Biophysical journal  87(3),  1507--1525
  (2004)

\bibitem{keener2009invariant}
Keener, J.P.: Invariant manifold reductions for markovian ion channel dynamics.
  Journal of Mathematical Biology  58(3),  447--457 (2009)

\bibitem{keener1998mathematical}
Keener, J.P., Sneyd, J.: Mathematical physiology, vol.~1. Springer (1998)

\bibitem{lagerstrom1972basic}
Lagerstrom, P., Casten, R.: Basic concepts underlying singular perturbation
  techniques. Siam Review  14(1),  63--120 (1972)

\bibitem{MATLAB:2013}
MATLAB: {version 1.7.0$\_$11 (R2013b)}. The MathWorks Inc., Natick,
  Massachusetts (2013)

\bibitem{murthy2012approximate}
Murthy, A., Islam, M.A., Bartocci, E., Cherry, E.M., Fenton, F.H., Glimm, J.,
  Smolka, S.A., Grosu, R.: Approximate bisimulations for sodium channel
  dynamics. In: Computational Methods in Systems Biology. pp. 267--287.
  Springer (2012)

\bibitem{NGVR12sasb}
Noel, V., Grigoriev, D., Vakulenko, S., Radulescu, O.: Tropical geometries and
  dynamics of biochemical networks application to hybrid cell cycle models. In:
  Feret, J., Levchenko, A. (eds.) Proceedings of the 2nd International Workshop
  on Static Analysis and Systems Biology ({SASB} 2011), Electronic Notes in
  Theoretical Computer Science, vol. 284, pp. 75--91. Elsevier (2012)

\bibitem{Noel2013a}
Noel, V., Grigoriev, D., Vakulenko, S., Radulescu, O.: Tropicalization and
  tropical equilibration of chemical reactions. In: Litvinov, G., Sergeev, S.
  (eds.) Tropical and Idempotent Mathematics and Applications, Contemporary
  Mathematics, vol. 616, pp. 261--277. American Mathematical Soc. (2014)

\bibitem{radulescu2012frontiers}
Radulescu, O., Gorban, A.N., Zinovyev, A., Noel, V.: {Reduction of dynamical
  biochemical reactions networks in computational biology}. Frontiers in
  Genetics  3(131) (2012)

\bibitem{radulescu2015symbolic}
Radulescu, O., Samal, S.S., Naldi, A., Grigoriev, D., Weber, A.: Symbolic
  dynamics of biochemical pathways as finite states machines. In: Computational
  Methods in Systems Biology. pp. 104--120. Springer (2015)

\bibitem{radulescu2015}
Radulescu, O., Vakulenko, S., Grigoriev, D.: Model reduction of biochemical
  reactions networks by tropical analysis methods. Mathematical Model of
  Natural Phenomena  10(3),  124--138 (2015)

\bibitem{samal2015geometric}
Samal, S.S., Grigoriev, D., Fr{\"o}hlich, H., Weber, A., Radulescu, O.: A
  geometric method for model reduction of biochemical networks with polynomial
  rate functions. Bulletin of mathematical biology  77(12),  2180--2211 (2015)

\bibitem{soliman2014constraint}
Soliman, S., Fages, F., Radulescu, O.: A constraint solving approach to model
  reduction by tropical equilibration. Algorithms for Molecular Biology  9(1),
  ~1 (2014)

\bibitem{suckley2003asymptotic}
Suckley, R., Biktashev, V.N.: {The asymptotic structure of the Hodgkin--Huxley
  equations}. International Journal of Bifurcation and Chaos  13(12),
  3805--3825 (2003)

\bibitem{tang1996simplification}
Tang, Y., Stephenson, J.L., Othmer, H.G.: {Simplification and analysis of
  models of calcium dynamics based on IP3-sensitive calcium channel kinetics.}
  Biophysical journal  70(1),  246 (1996)

\bibitem{tikhonov1952systems}
Tikhonov, A.N.: {Systems of differential equations containing small parameters
  in the derivatives}. Matematicheskii sbornik  73(3),  575--586 (1952)

\bibitem{wechselberger2013canard}
Wechselberger, M., Mitry, J., Rinzel, J.: Canard theory and excitability. In:
  Nonautonomous dynamical systems in the life sciences, pp. 89--132. Springer
  (2013)

\end{thebibliography}





	


\end{document}
