\begin{abstract}
We present a bifurcation analysis of electrical alternans in 
the two-current Mitchell-Schaeffer (MS) cardiac-cell model using
the theory of $\delta$-decidability over the reals.  Electrical 
alternans is a phenomenon characterized by a variation in the 
successive \emph{Action Potential Durations} (APDs) generated 
by a single cardiac cell or tissue.  Alternans are known to 
initiate re-entrant waves and are an important physiological 
indicator of an impending life-threatening arrhythmia such 
as ventricular fibrillation.  The bifurcation analysis we
perform determines, for each control parameter $\tau$ of
the MS model, the \emph{bifurcation point} in the range of $\tau$
such that a small perturbation to this value results in a transition
from alternans to non-alternans behavior.  To the best of our
knowledge, our analysis represents the first formal verification of
non-trivial dynamics in a numerical cardiac-cell model.

Our approach to this problem rests on encoding alternans-like
behavior in the MS model as a 11-mode, multinomial hybrid automaton
(HA).  For each model parameter, we then apply a sophisticated,
guided-search-based reachability analysis to this HA to estimate
parameter ranges for both alternans and non-alternans behavior.
The bifurcation point separates these two ranges, but with an 
uncertainty region due to the underlying $\delta$-decision procedure.
This uncertainty region, however, can be reduced by decreasing $\delta$
at the expense of increasing the model exploration time.  Experimental 
results are provided that highlight the effectiveness of this method.
\end{abstract}