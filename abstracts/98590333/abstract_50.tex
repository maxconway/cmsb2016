% This is LLNCS.DEM the demonstration file of
% the LaTeX macro package from Springer-Verlag
% for Lecture Notes in Computer Science,
% version 2.4 for LaTeX2e as of 16. April 2010
%
\documentclass{llncs}
%
\usepackage{makeidx}  % allows for indexgeneration
 \usepackage{amsmath}
 \usepackage{graphicx}
%
\begin{document}
%
\frontmatter          % for the preliminaries
%
\pagestyle{headings}  % switches on printing of running heads
\addtocmark{Revealing Biomarker Mixtures in Lipid Pools from large-scale Lipidomics} % additional mark in the TOC

%

%
\mainmatter              % start of the contributions
%
\title{Revealing Biomarker Mixtures in Lipid Pools from large-scale Lipidomics}
%
\titlerunning{Revealing Biomarker Mixtures in Lipid Pools from large-scale Lipidomics}  % abbreviated title (for running head)
%                                     also used for the TOC unless
%                                     \toctitle is used
%
\author{Kai Loell\inst{1},  Albert Koulman\inst{2}  James Smith \inst{1,2,3}\\ 
\texttt{10311kai@gmail.com, ak675@cam.ac.uk, j.smith252@leeds.ac.uk}}
\authorrunning{Kai et al.} % abbreviated author list (for running head)
%
%%%% list of authors for the TOC (use if author list has to be modified)
%\tocauthor{Kai Loell, Albert Koulman, and James Smith}
%
\institute{Cambridge Computational Biology Institute,  
Department of Applied Mathematics and Theoretical Physics, Centre for Mathematical Sciences, University of Cambridge, Wilberforce Rd, Cambridge CB3 0WA, U.K.
\and
MRC Elsie Widdowson Laboratory, (Formerly) MRC Human Nutrition Research, 120 Fulbourn Rd, Cambridge CB1 9NL, U.K.
\and
School of Food Science and Nutrition, Faculty of Mathematics and Physical Sciences, University of Leeds, Leeds LS2 9JT, U.K.
}
\maketitle              % typeset the title of the contribution
\vbox{}
\begin{abstract}
Lipids are key structural elements, energy sources, and components for intracellular signalling and metabolic processes. 
Their constituents are a small number of fatty acids (FAs), indicators of metabolic health and nutrition and biomarkers for disease risk. 
Lipids contain singlet, doublet and triplet combinations of FAs. Different combinations of FAs can have equivalent configurations that in aggregate are observed as lipid mixture pools. 
Traditionally, the lipid pools have been considered to be biomarkers, however it is now recognised that sub-populations of explicit lipid species are more informative. 
Lipid biomarkers for metabolic states, so far, come from the latent (hidden) structure of sub-populations in the pools and this needs to be addressed.

Epidemiological high-resolution lipidomics data is required to derive the \textit{mixtures} of lipid species contributing to lipid pools. 
FA signals are acquired using gas chromatography and lipid profiling performed by direct infusion high resolution mass spectrometry. 
Profiling identifies lipid mixture pools as spectral peaks separated by their $m/z$ ratio. However, not all constituent lipid sub-populations are easily distinguished. 
Furthermore, the data generated is \textit{compositional} with the signals of FAs and lipid pools normalised separately.

Our approach to this problem uses both lipid pool and FA data. Lipid data is re-scaled to account for the combinations of FAs required in each observed pool. 
A linear algebra Gauss-Jordan reduction algorithm is applied to the stoichiometry of FAs incorporated in the explicit lipid species and the combinations of lipid species in every pool.  
The method solves the contributing lipid species sub-populations, that is, the representative combinations of FAs that form the pools. 
Abundances of explicit FA combinations not only improve lipid biomarker identification but also provide a more detailed picture of metabolic responses.

\begin{keywords}compositional mixture modelling, optimisation, biomarkers, metabolic states, big data, lipidomics \end{keywords}
\end{abstract}
\end{document}
