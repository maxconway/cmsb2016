% This is LLNCS.DEM the demonstration file of
% the LaTeX macro package from Springer-Verlag
% for Lecture Notes in Computer Science,
% version 2.4 for LaTeX2e as of 16. April 2010
%
\documentclass{llncs}
%
\usepackage{makeidx}  % allows for indexgeneration
\usepackage{amsmath}  % maths
\usepackage{graphicx} % graphics
\usepackage{color} % colour for annotation and comments
%
\begin{document}
%
\frontmatter          % for the preliminaries
%
\pagestyle{headings}  % switches on printing of running heads
\addtocmark{Game Theoretic Consideration of Transgenic Bacteria in the Human Gut Microbiota Converting Omega-6 to Omega-3 Fats} % additional mark in the TOC

%

%
\mainmatter              % start of the contributions
%
\title{Game Theoretic Consideration of Transgenic Bacteria in the Human Gut Microbiota Converting Omega-6 to Omega-3 Fats}
%
\titlerunning{Game Theoretic Consideration of Transgenic Bacteria in the Human Gut Microbiota Converting Omega-6 to Omega-3 Fats}  % abbreviated title (for running head)
%                                     also used for the TOC unless
%                                     \toctitle is used
%
\author{Ahmed M. Ibrahim \inst{1} \and James Smith \inst{2, 3, 4}\\
\texttt{wetawdt@gmail.com, j.smith252@leeds.ac.uk}}
\authorrunning{Ibrahim and Smith} % abbreviated author list (for running head)
%
%%%% list of authors for the TOC (use if author list has to be modified)
%\tocauthor{Ahmed M.Ibrahim, James Smith}
%
\institute{44 El-Geish St, Mansoura, Dakahlia, Egypt.
\and
Cambridge Computational Biology Institute, Department of Applied Mathematics and Theoretical Physics, Centre for Mathematical Sciences, Wilberforce Rd, Cambridge CB3 0WA, U.K.
\and
MRC Elsie Widdowson Laboratory, (Formerly) MRC Human Nutrition Research, 120 Fulbourn Rd, University of Cambridge, Cambridge CB1 9NL, U.K.
\and
School of Food Science and Nutrition, Faculty of Mathematics and Physical Sciences, University of Leeds, Leeds LS2 9JT, U.K.}
\maketitle              % typeset the title of the contribution
%\vbox{}
\begin{abstract}
Prophylactic use of functional foods and the design of nutraceuticals has a far-reaching public health benefit. 
Understanding the  phenotypic manipulations needed to take advantage of gut microbial ecology is fundamental to bioengineering and the food, diet and health industries. 
This work considers a hypothetical adjustment of gut microbiota by an introduced transgenic bacterial strain that 
contributes to increased exposure of essential omega-3 (n-3) poly-unsaturated fatty acids, the so-called \textit{fish oils}. 
Absorption of the essential poly-unsaturated fats from food is dominated by the 
omega-6 (n-6)  fats over the omega-3 (n-3) fats. Unfortunately, long-term depleted levels of
n-3-containing lipids in blood plasma is a high-risk indicator for outcomes such as metabolic syndrome, cardiovascular disease and diabetes-related conditions. 

In our vignette, a genetically modified strain converts excessive dietary n-6 into bioavailable n-3 in the gut. 
Maintaining a long-term co-existence between indigenous gut bacteria and the transgenic strain is the challenge.   
Game theory is an appropriate formalism for exploring such conflicts. 
We show that long-term co-existence is predicted if the two forms of bacteria engage in the Snowdrift game.  
Our model explores putative mechanisms for addressing metabolic syndrome and related conditions 
by locally increasing n-3 production by the transgenic gut bacteria. 
Our model suggests long-term therapeutic supplementation by a functional probiotic food is possible without detriment to the indigenous bacteria. 
\end{abstract}
\begin{keywords} game theory, snow drift game, prisoners' dilemma, non-linear behaviour, gut microbiome, fat metabolism.\end{keywords}

\end{document}
